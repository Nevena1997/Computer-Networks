\documentclass[]{article}

\usepackage[margin=0.7in]{geometry}
\usepackage{url}
\usepackage{float}
\usepackage[graphicx]{realboxes}
\usepackage{listings}
\usepackage{textcomp}
\usepackage{xcolor}
\usepackage{adjustbox}
\lstset {
    language=HTML,
    frame=none,
    %xleftmargin=-.25in,
    %xrightmargin=.25in
    framesep=10pt,
    tabsize=4,
    showstringspaces=false,
    upquote=true,
    commentstyle=\color{black},
    keywordstyle=\color{black},
    stringstyle=\color{black},
    basicstyle=\small\ttfamily,
    emph={int,char,double,float,unsigned,void,bool},
    emphstyle={\color{black}},
    escapechar=\&,
    classoffset=1,
    morekeywords={>,<,.,;,,,-,!,=,~},
    keywordstyle=\color{black},
    classoffset=0,
    breaklines=true
}
\pagenumbering{gobble}

\title{Ra\v{c}unarske mre\v{z}e 2019, Ispit - Januar 1, 4I}
\author{}
\date{16.01.2019.}

\begin{document}
\maketitle

\begin{enumerate}
  \item Sockets and Channels \textbf{(15p)}
  \begin{itemize}
    \item Napraviti Java aplikaciju koja ima ulogu klijenta. Povezati se na lokalni server na portu 12345 koriste\'c{}i Java Sockets API i u beskona\v{c}noj petlji slati serveru niske koje korisnik unosi sa standardnog ulaza sve dok se ne unese niska \texttt{stop}. \hfill (4p)
    \item Napraviti Java aplikaciju koja ima ulogu servera. Pokrenuti \textbf{neblokiraju\'c{}i} lokalni server na portu 12345. Kada klijent po\v{s}alje serveru nisku \texttt{start}, server klijentu \v{s}alje sekvencu od 20 nasumi\v{c}no odabranih karaktera od A-Z (ta sekvenca se nasumi\v{c}no bira kad je klijent zatra\v{z}i). Klijent ispisuje dobijenu sekvencu na standardni izlaz. \hfill (8p)
    \item Omogu\'c{}iti da server tokom rada neprestano ispisuje broj povezanih klijenata. \hfill (2p)
    \item Postarati se da su svi resursi ispravno zatvoreni u slu\v{c}aju izuzetka. \hfill (1p)
  \end{itemize}

  \item Swing \textbf{(15p)}
  Napraviti Java Swing aplikaciju koja ima ulogu jednostavnog Web browser-a. Izgled aplikacije je dat na slici ispod. Potrebno je ispo\v{s}tovati izgled aplikacije (ne me\v{s}ati redosled komponenti i postarati se da su odnosi u veli\v{c}ini kao na slici). Primeri su na slede\'c{}oj strani.
  \begin{itemize}
    \item Napraviti prozor i u njega dodati skrolabilnu komponentu za prikaz sadr\v{z}aja web stranice. \hfill (2p)
    \item Omogu\'c{}iti da je pra\'c{}enje linkova podr\v{z}ano - kada se klikne na link da se promeni sadr\v{z}aj komponente za prikaz na sadr\v{z}aj web stranice na koju link pokazuje (za testiranje napraviti par lokalnih HTML fajlova sa linkovima izmedju njih). \hfill (2p)
    \item Dodati dugme \texttt{Undo} (na slici ozna\v{c}eno karakterom \texttt{<}) na \v{c}iji klik se sadr\v{z}aj komponente za prikaz vra\'c{}a na sadr\v{z}aj prethodne web-strane koja je bila prikazana (ukoliko postoji). \hfill (3p)
    \item Dodati dugme \texttt{Redo} (na slici ozna\v{c}eno karakterom \texttt{>}) na \v{c}iji klik se sadr\v{z}aj komponente za prikaz vra\'c{}a na sadr\v{z}aj naredne web-strane koja je bila prikazana pre Undo operacije (ukoliko postoji). \hfill (2p)
    \item Dodati dugme \texttt{ci} \v{c}iji je efekat da ukloni sve slike iz komponente za prikaz. Preciznije, na klik ovog dugmeta se HTML sadr\v{z}aj komponente za prikaz \v{c}isti od svih \texttt{img} tagova zajedno sa njihovim sadr\v{z}ajem. Tako dobijeni filtrirani HTML prikazati u komponenti za prikaz. \hfill (6p)
  \end{itemize}

  \begin{figure}[H]
    \centering
    \includegraphics[scale=0.5]{fig2.PNG}
    \label{fig2}
  \end{figure}

\end{enumerate}

\newpage

HTML fajlovi za zadatak 2:

\begin{itemize}
  \item 1.html
    \begin{lstlisting}
      <!DOCTYPE html>
        <html>
          <body>
            <h1>1.HTML</h1>

            <p>My paragraph.</p>
            <a href="2.html">2</a> 

            <img src="SOME_IMAGE_URL" alt="no image"/>

            <p>My paragraph.</p>
          </body>
        </html>
    \end{lstlisting}
  \item 2.html
    \begin{lstlisting}
      <!DOCTYPE html>
        <html>
          <body>
            <h1>2.HTML</h1>
  
            <a href="1.html">1</a> 
            <a href="3.html">3</a> 
            <p>My paragraph 2.</p>
          </body>
        </html>
    \end{lstlisting}
  \item 3.html
     \begin{lstlisting}
      <!DOCTYPE html>
        <html>
          <body>
            <h1>3.HTML</h1>
    
            <a href="1.html">1</a> 
            <a href="2.html">2</a> 
            <p>My paragraph 3.</p>
          </body>
        </html>
    \end{lstlisting}
\end{itemize}


\end{document}
