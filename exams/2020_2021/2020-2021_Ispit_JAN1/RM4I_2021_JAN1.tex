\documentclass[]{article}

% ----------- BEGIN PACKAGES -----------
\usepackage{verbatim}
\usepackage[utf8]{inputenc}
\usepackage[english,serbian]{babel}
\usepackage[margin=0.7in]{geometry}
\usepackage{url}
\usepackage{float}
\usepackage[graphicx]{realboxes}
\usepackage{listings}
\usepackage{textcomp}
\usepackage{xcolor}
\usepackage{titlesec}
\usepackage{adjustbox}
\lstset {
    language=Java,
    frame=none,
    %xleftmargin=-.25in,
    %xrightmargin=.25in
    framesep=10pt,
    tabsize=4,
    showstringspaces=false,
    upquote=true,
    commentstyle=\color{black},
    keywordstyle=\color{black},
    stringstyle=\color{black},
    basicstyle=\small\ttfamily,
    emph={int,char,double,float,unsigned,void,bool},
    emphstyle={\color{black}},
    escapechar=\&,
    classoffset=1,
    morekeywords={>,<,.,;,,,-,!,=,~},
    keywordstyle=\color{black},
    classoffset=0,
    breaklines=true
}
\pagenumbering{gobble}
% ----------- END PACKAGES -----------

% ----------- BEGIN PREAMBLE -----------
\titlespacing\title{left spacing}{before spacing}{after spacing}[right]

\title{Računarske mreže, Ispit - JAN1}
\date{18.01.2021.}

\begin{document}

\makeatletter
\begin{center}

{\fontsize{12pt}{14pt}\selectfont\bfseries\@title\par}
\@date
\vspace{5mm}

\noindent\fbox{%
    \parbox{\textwidth}{%
      Pro\v{c}itati sve zadatke \textbf{pa\v{z}ljivo} pre rada - sve \v{s}to nije navedeno ne mora da se implementira! 

      Na \texttt{Desktop}-u se nalazi zip arhiva. Unutar arhive se nalazi direktorijum u formatu \texttt{rm\_rok\_Ime\_Prezime\_mXGGXXX}\\
      u kome se nalazi validan IntelliJ projekat. Izvu\'c{}i direktorijum iz arhive na Desktop i ubaciti svoje podatke u ime.\\
      Otvoriti IntelliJ IDEA, izabrati opciju \texttt{Open project} (ne \texttt{Import project}!) i otvoriti pomenuti direktorijum.\\ 
      Sve kodove ostaviti unutar ve\'c{} kreiranih Java fajlova. \textbf{Kodovi koji se ne prevode se ne\'c{}e pregledati.}\\
      \textbf{Nepo\v{s}tovanje formata ulaza/izlaza nosi kaznu od -10\% poena na zadatku!}
      Vreme za rad: \textbf{3h}.
    }%
}
\end{center}
\makeatother
% ----------- END PREAMBLE -----------

\vspace{5pt}

% ----------- BEGIN DOCUMENT -----------
\begin{enumerate}

% ----------- BEGIN 1 -----------
  \item \textbf{Čitamo poeziju (15p) (za studente koji nisu radili projekat)}
  \\Napisati program koji na standardni izlaz ispisuje informacije o pesmama.
  \begin{itemize}
	\item Sa standardnog ulaza se unosi reč. U direktorijumu \texttt{pesme} unutar direktorijuma \texttt{tests} na Desktop-u se nalaze datoteke koje sadrže pesme (naziv datoteke je naslov pesme, a svaka linija je jedan stih). 
	
     Na standardni izlaz, za svaku datoteku, ispisati naziv pesme, ispisati najduži stih i koliko se puta reč pojavljuje u toj datoteci. \hfill (6p)
    \item Za svaku datoteku pokrenuti zasebnu nit koja će obrađivati i ispisivati informacije. \hfill (3p)
    \item Postarati se da se ispisi svake niti na standardni izlaz ne prepliću. \hfill (4p)
    \item Postarati se da program ispravno obrađuje specijalnim slučajeve (npr. ako datoteka ne postoji na datoj putanji) i ispravno zatvoriti sve korišćene resurse u slučaju izuzetka. \hfill (2p)
  \end{itemize}
  
\noindent
\begin{tabular}{l}
\begin{lstlisting}
Spisak datoteka:
	\home\ispit\Desktop\tests\pesme\Dolap.txt
	\home\ispit\Desktop\tests\pesme\DomovinaSeBraniLepotom.txt
	\home\ispit\Desktop\tests\pesme\MozdaSpava.txt
	\home\ispit\Desktop\tests\pesme\LjubavnaPesma.txt
	\home\ispit\Desktop\tests\pesme\AlSuToGrdneMuke.txt
	\home\ispit\Desktop\tests\pesme\PoslednjaPesma.txt
\end{lstlisting}
\\
\begin{lstlisting}
Ulaz:	kao 
\end{lstlisting}
\\
\begin{lstlisting}
Izlaz:
	Dolap
	Pusti snovi! Napred, vrance, nemoj stati,
	4
	DomovinaSeBraniLepotom
	Sestrinom suzom majcinom brigom
	0
	MozdaSpava
	Ja sad nemam svoju dragu, i njen ne znam glas.
	3
	LjubavnaPesma
	I sa mnom ljube, ceznu, strepe, zude!
	0
	AlSuToGrdneMuke
	kad bi mogle da se pruze!
	0
	PoslednjaPesma
	Njen je plast suncani mozda tkivo lazi;
	2
\end{lstlisting}
\end{tabular}
% ----------- END 1 -----------
\vspace{15pt}

\begin{center}
  \textbf{------------------------------------------------------------------------------------------------------------------------------}
\end{center}
\textit{Napomena: Ohrabrujemo studente da koriste \texttt{netcat} kako bi testirali delimi\v{c}ne implementacije i otkrili gre\v{s}ke pre vremena. Takodje, ukoliko se npr. presko\v{c}i implementacija servera, mo\v{z}e se mock-ovati server putem \texttt{netcat}-a.} 
\begin{center}
  \textbf{--------------------------------------------------- Okrenite stranu! ---------------------------------------------------}
\end{center}

\newpage

% ----------- BEGIN 2 -----------
\item \textbf{TCP Sockets (25p/18p)}
\\Napraviti Java klijent-server TCP igru iks-oks. 
\begin{itemize}
  \item Kreirati Java aplikaciju koja ima ulogu iks-oks igra\v{c}a - klijenta. Klijenti se povezuju na lokalni server na portu $12345$ i periodi\v{c}no izvr\v{s}avaju naredne operacije sve dok server ne signalizira kraj igre:
  \hfill (3p)
  \begin{itemize}
    \item ispis trenutnog stanja igre tj.~iks-oks table (proizvoljno implementirati)
    \item ucitavanje poteza od strane igraca (potez proizvoljno implementirati)
    \item slanje poteza serveru (nije potrebno implementirati validaciju poteza na klijentskoj strani)
  \end{itemize}
  \item Kreirati Java aplikaciju koriste\'c{}i TCP Sockets API koja ima ulogu iks-oks servera. Server nakon pokretanja najpre \v{c}eka da se dva klijenta pove\v{z}u a zatim pokre\'c{}e igru. Nakon zavr\v{s}etka igre, server ponovo \v{c}eka na igra\v{c}e i nakon toga zapo\v{c}inje novu igru. \hfill (4p)
  \item Implementirati iks-oks igru na strani servera. Naizmeni\v{c}no \v{c}itati poteze igra\v{c}a, a\v{z}urirati stanje igre i poslati a\v{z}urirano stanje klijentima. Kad se igra zavr\v{s}i, poslati igra\v{c}ima informaciju da se igra zavr\v{s}ila i prekinuti vezu sa igra\v{c}ima. \hfill (9p)
  \item U slu\v{c}aju da igra\v{c} po\v{s}alje nevalidan potez serveru, server odgovara nazad porukom \texttt{Nevalidan potez}, i o\v{c}ekuje ponovo potez od istog igra\v{c}a. \hfill (2p) 
  \item Postarati se da su svi resursi ispravno zatvoreni u slu\v{c}aju izuzetka. \hfill (2p)
\end{itemize}

\vspace{10pt}
\noindent
\begin{tabular}{c c}
  \begin{lstlisting}
(klijent 1)       |
                  |
    ---           |
    ---           |
    ---           |
                  |
    > 5           |
    --O           |
    -X-           |
    ---           |
                  |
    > 6           |
    --O           |
    OXX           |
    ---           |
                  |
    > 2           |
    OXO           |
    OXX           |
    ---           |
                  |
    > 8           |
    OXO           |
    OXX           |
    -X-           |
  \end{lstlisting}&
  \begin{lstlisting}
|      (klijent 2)
|
|          ---
|          -X-
|          ---
|
|          > 3
|          --O
|          -XX
|          ---
|
|          > 4
|          --O
|          OXX
|          -X-
|
|          > 1
|          OXO
|          OXX
|          -X-
|           
|
|
|
  \end{lstlisting}
\end{tabular}
% ----------- END 2 -----------

\vspace{15pt}

% ----------- BEGIN 3 -----------
\item \textbf{UDP Sockets (20p/12p)}
\\Napraviti UDP server koji ra\v{c}una povr\v{s}ine krugova \v{c}ije polupre\v{c}nike \v{s}alju klijenti.
\begin{itemize}
  \item Napraviti Java klasu koja ima ulogu UDP klijenta koriste\'c{}i Java Datagram API. Klijent \v{s}alje datagram serveru na portu $31415$ u kome se nalazi jedan realan broj u\v{c}itan sa standardnog ulaza. Primiti i ispisati odgovor servera na standardni izlaz. \hfill (9p/5p)
  \item Napraviti Java klasu koja ima ulogu lokalnog UDP servera koji oslu\v{s}kuje na portu $31415$, koriste\'c{}i Java \texttt{UDP} API. Nakon primanja datagrama, server odgovara klijentu realnim brojem koji predstavlja povr\v{s}inu kruga \v{c}iji je polupre\v{c}nik jednak broju koji je klijent poslao. Ukoliko server primi negativan broj od klijenta, klijentu odgovoriti porukom \texttt{Neispravan poluprecnik}. \hfill (10p/6p)
  \item Postarati se da su svi resursi ispravno zatvoreni u slu\v{c}aju izuzetka. \hfill (1p)
\end{itemize}
% ----------- END 3 -----------

\end{enumerate}

\end{document}
% ----------- END DOCUMENT -----------
