\documentclass[]{article}

% ----------- BEGIN PACKAGES -----------
\usepackage[utf8]{inputenc}
\usepackage[english,serbian]{babel}
\usepackage[margin=0.7in]{geometry}
\usepackage{url}
\usepackage{float}
\usepackage[graphicx]{realboxes}
\usepackage{listings}
\usepackage{textcomp}
\usepackage{xcolor}
\usepackage{titlesec}
\usepackage{adjustbox}
\lstset {
    language=Java,
    frame=none,
    %xleftmargin=-.25in,
    %xrightmargin=.25in
    framesep=10pt,
    tabsize=4,
    showstringspaces=false,
    upquote=true,
    commentstyle=\color{black},
    keywordstyle=\color{black},
    stringstyle=\color{black},
    basicstyle=\small\ttfamily,
    emph={int,char,double,float,unsigned,void,bool},
    emphstyle={\color{black}},
    escapechar=\@,
    classoffset=1,
    morekeywords={>,<,.,;,,,-,!,=,~},
    keywordstyle=\color{black},
    classoffset=0,
    breaklines=true
}
\pagenumbering{gobble}
% ----------- END PACKAGES -----------

% ----------- BEGIN PREAMBLE -----------
\titlespacing\title{left spacing}{before spacing}{after spacing}[right]

\title{Ra\v{c}unarske mre\v{z}e, Ispit - SEP2 2021}
\date{}

\begin{document}

\makeatletter
\begin{center}

{\fontsize{12pt}{14pt}\selectfont\bfseries\@title\par}
\@date
\vspace{5mm}

\noindent\fbox{%
    \parbox{\textwidth}{%
      Pro\v{c}itati sve zadatke \textbf{pa\v{z}ljivo} pre rada - sve \v{s}to nije navedeno ne mora da se implementira!\\
      Na \texttt{Desktop}-u se nalazi zip arhiva. Unutar arhive se nalazi IntelliJ projekat u formatu \texttt{rm\_rok\_Ime\_Prezime\_mXGGXXX}.\\ Dekompresovati arhivu na Desktop i ubaciti svoje podatke u ime pomenutog direktorijuma.\\
      Otvoriti IntelliJ IDEA, izabrati opciju \texttt{Open project} (\textbf{ne} \texttt{Import project}!) i otvoriti pomenuti \textbf{direktorijum}.\\ 
      Sve kodove ostaviti unutar ve\'c{} kreiranih Java fajlova. \textbf{Kodovi koji se ne prevode se ne\'c{}e pregledati.}\\
      \textbf{Nepo\v{s}tovanje formata ulaza/izlaza nosi kaznu od -10\% poena na zadatku!}
    }%
}
\end{center}
\makeatother
% ----------- END PREAMBLE -----------

\vspace{5pt}

% ----------- BEGIN DOCUMENT -----------
\begin{enumerate}

% ----------- BEGIN 2 -----------
\item \textbf{TCP Sockets (25p/18p)}
\\Napraviti osnovu za TCP klijent-server Java aplikaciju koja poma\v{z}e u odr\v{z}avanju server farme. Odr\v{z}avanje server farme podrazumeva obilazak i obradu svakog ra\v{c}unara. Takodje, u slučaju kvara, neophodno je ispraviti odgovaraju\'c{}i ra\v{c}unar. Po\v{s}to je broj ra\v{c}unara u farmi veliki, potrebno je puno tehni\v{c}ara i sistem za pra\'c{}enje trenutnog stanja farme. Sistem bi trebalo da \v{c}uva status za svaki od ra\v{c}unara u mre\v{z}noj topologiji farme i da omogu\'c{}i tehni\v{c}arima da udaljeno promene statuse ra\v{c}unara. Odlu\v{c}eno je da se za ove potrebe kreira Java serverska aplikacija koja \'c{}e omogu\'c{}iti tehni\v{c}arima da se pove\v{z}u i postavljaju upite sistemu --- kakvo je trenutno stanje farme, markiraj ra\v{c}unar \texttt{X} kao pokvaren, itd. Topologija serverske farme je takva da se mo\v{z}e predstaviti matricom veli\v{c}ine \texttt{NxN} (izgledno je da \'c{}e se veli\v{c}ina menjati u budu\'c{}nosti) i da se svaki ra\v{c}unar mo\v{z}e "adresirati" parom brojeva koji predstavljaju vrstu i kolonu u matrici, redom.

\begin{itemize}
  \item Napraviti Java klasu koja ima ulogu lokalnog TCP servera koji osluškuje na portu $13370$. Pre pokretanja servera, zatra\v{z}iti od korisnika da unese broj \texttt{N} koji predstavlja veli\v{c}inu matrice server farme. Nakon toga, server oslu\v{s}kuje na pomenutom portu i svakom novom klijentu inicijalno \v{s}alje stanje serverske farme --- trenutno stanje matrice farme. Ispisati matricu kao u primeru ispod --- sa \texttt{o} su ozna\v{c}eni ispravni, a sa \texttt{x} neispravni ra\v{c}unari. Prilikom pokretanja servera pretpostaviti da su svi ra\v{c}unari ispravni. Server svakog klijenta treba da obradi u zasebnoj niti. \hfill (4p/2p)
  \item Napraviti Java klasu koja ima ulogu lokalnog TCP klijenta. Nakon uspostavljanja konekcije sa serverom na portu $12345$, klijent čeka na trenutno stanje farme od servera i ispisuje ga na standardni izlaz. Nakon toga, klijent \v{s}alje serveru upite koje tehni\v{c}ar unosi sa standardnog ulaza, sve dok se ne unese upit \texttt{exit}, nakon \v{c}ega klijent prestaje s radom. \hfill (3p/2p)
  \item Nakon slanja stanja farme klijentu, server \v{c}eka na upite klijenta. Upiti mogu biti:  \hfill (13/11p)
  \begin{enumerate}
    \item \texttt{list} --- odgovor servera treba da bude trenutno stanje farme
    \item \texttt{mark X Y} --- markira server sa koordinatama \texttt{(X, Y)} kao neispravan i \v{s}alje klijentu a\v{z}urirano stanje farme
    \item \texttt{repair X Y} --- uklanja marker neispravnosti serveru sa koordinatama \texttt{(X, Y)} i \v{s}alje klijentu a\v{z}urirano stanje farme
    \item \texttt{exit} --- server prestaje da obradjuje klijenta i raskida vezu (bez slanja odgovora klijentu)
  \end{enumerate}
  \textit{(ukoliko koordinate nisu ispravne, server ne vr\v{s}i nikakve modifikacije)}
  \item Sinhronizovati pristup modelu farme na serverskoj strani. \hfill (3p/2p)
  \item Postarati se da su svi resursi ispravno zatvoreni u slučaju izuzetka. \hfill (2p/1p)
\end{itemize}

\vspace{10pt}

\noindent
\begin{tabular}{lll}
\begin{lstlisting}
$ java Client.java
+|01234
++-----
0|ooooo
1|ooooo
2|ooooo
3|ooooo
4|ooooo
\end{lstlisting}&
\begin{lstlisting}
(upit) mark 3 3
+|01234
++-----
0|ooooo
1|ooooo
2|ooxoo // markirano polje (3,3)
3|oooox // drugi tehnicar markirao
4|ooooo
\end{lstlisting}&
\begin{lstlisting}
(upit) repair 3 3
+|01234
++-----
0|oooxo
1|ooooo
2|ooooo // ispravljeno
3|oooox
4|ooooo
\end{lstlisting}
\end{tabular}
% ----------- END 2 -----------

\vspace{15pt}

\begin{center}
  \textbf{------------------------------------------------------------------------------------------------------------------------------}
\end{center}
\textit{Napomena: Ohrabrujemo studente da koriste \texttt{netcat} kako bi testirali delimi\v{c}ne implementacije i otkrili gre\v{s}ke pre vremena. Takodje, ukoliko se npr. presko\v{c}i implementacija servera, mo\v{z}e se mock-ovati server putem \texttt{netcat}-a.} 
\begin{center}
  \textbf{--------------------------------------------------- Okrenite stranu! ---------------------------------------------------}
\end{center}

\newpage

% ----------- BEGIN 3 -----------
\item \textbf{UDP Sockets - Cezarova šifra (20p/12p)}
\\Napraviti klijent-server Java aplikaciju, koristeći \texttt{Datagram API}, koja će služiti za šifrovanje poruka.
\begin{itemize}
  \item Napisati Java klasu koja ima ulogu lokalnog UDP klijenta. Klijent sa standardnog ulaza učitava rečenice koje želi da šifruje, sve dok ne unese \textit{KRAJ}. Za svaku učitanu rečenicu, klijent šalje serveru paket sa rečenicom kao sadržajem. Za svaki poslati paket, klijent dobija paket nazad od servera, čiji sadržaj ispisuje na standardni izlaz. \hfill (6p/4p)
  \item Napisati Java klasu koja ima ulogu lokalnog UDP servera koji osluškuje na portu 12345. Server prihvata pakete od klijenata i za svaki prihvaćen paket ispisuje poruku na standardni izlaz sa informacijama o rednom broju primljenog paketa i portu sa kog je paket stigao. \hfill (6p/3p)
  \item Za svaki prihvaćen paket, server šifruje rešenicu i šalje je nazad klijentu. Najpre slučajnim izborom generiše ključ, jedan ceo broj u intervalu [0, 25], koji ispisuje na standardni izlaz, a zatim svaki karakter rečenice zamenjuje karakterom iz alfabeta ciklično pomerenim za broj mesta određenim vrednošću ključa. Na primer, za vrednost ključa jednak 3, \textit{a} će biti zamenjeno karakterom \textit{d}, a \textit{z} karakterom \textit{c}. Pretpostaviti da će sve rečenice biti napisane svim malim slovima. \hfill (6p/4p)
  \item Postarati se da su svi resursi ispravno zatvoreni u slučaju izuzetka. \hfill (2p/1p)
\end{itemize}
% ----------- END 3 -----------

\vspace{15pt}

% ----------- BEGIN 1 -----------
\item \textbf{Nesto (15p) (za studente koji nisu radili projekat)}

% ----------- END 1 -----------

\end{enumerate}

\end{document}
% ----------- END DOCUMENT -----------
