\documentclass[]{article}

% ----------- BEGIN PACKAGES -----------
\usepackage[utf8]{inputenc}
\usepackage[english,serbian]{babel}
\usepackage[margin=0.7in]{geometry}
\usepackage{url}
\usepackage{float}
\usepackage[graphicx]{realboxes}
\usepackage{listings}
\usepackage{textcomp}
\usepackage{xcolor}
\usepackage{titlesec}
\usepackage{adjustbox}
\lstset {
    language=Java,
    frame=none,
    %xleftmargin=-.25in,
    %xrightmargin=.25in
    framesep=10pt,
    tabsize=4,
    showstringspaces=false,
    upquote=true,
    commentstyle=\color{black},
    keywordstyle=\color{black},
    stringstyle=\color{black},
    basicstyle=\small\ttfamily,
    emph={int,char,double,float,unsigned,void,bool},
    emphstyle={\color{black}},
    escapechar=\&,
    classoffset=1,
    morekeywords={>,<,.,;,,,-,!,=,~},
    keywordstyle=\color{black},
    classoffset=0,
    breaklines=true
}
\pagenumbering{gobble}
% ----------- END PACKAGES -----------

% ----------- BEGIN PREAMBLE -----------
\titlespacing\title{left spacing}{before spacing}{after spacing}[right]

\title{Ra\v{c}unarske mre\v{z}e, Ispit - JAN2}
\date{04.02.2021.}

\begin{document}

\makeatletter
\begin{center}

{\fontsize{12pt}{14pt}\selectfont\bfseries\@title\par}
\@date
\vspace{5mm}

\noindent\fbox{%
    \parbox{\textwidth}{%
      Pro\v{c}itati sve zadatke \textbf{pa\v{z}ljivo} pre rada - sve \v{s}to nije navedeno ne mora da se implementira! 

      Na \texttt{Desktop}-u se nalazi zip arhiva. Unutar arhive se nalazi direktorijum u formatu \texttt{rm\_rok\_Ime\_Prezime\_mXGGXXX}\\
      u kome se nalazi validan IntelliJ projekat. Izvu\'c{}i direktorijum iz arhive na Desktop i ubaciti svoje podatke u ime.\\
      Otvoriti IntelliJ IDEA, izabrati opciju \texttt{Open project} (ne \texttt{Import project}!) i otvoriti pomenuti direktorijum.\\ 
      Sve kodove ostaviti unutar ve\'c{} kreiranih Java fajlova. \textbf{Kodovi koji se ne prevode se ne\'c{}e pregledati.}\\
      \textbf{Nepo\v{s}tovanje formata ulaza/izlaza nosi kaznu od -10\% poena na zadatku!}
      Vreme za rad: \textbf{3h}.
    }%
}
\end{center}
\makeatother
% ----------- END PREAMBLE -----------

\vspace{5pt}

% ----------- BEGIN DOCUMENT -----------
\begin{enumerate}

% ----------- BEGIN 1 -----------
\item \textbf{Log-file parser (15p) (za studente koji nisu radili projekat)}
\\Napraviti Java aplikaciju koja selektivno parsira log fajl.
\begin{itemize}
  \item Napraviti Java aplikaciju koja prima putanju do regularnog fajla koji predstavlja log fajl i \v{c}ita ga koriste\'c{}i URL klasu i FILE protokol. Ispisati sadr\v{z}aj fajla na standardni izlaz. \hfill (3p)
  \item U fajlu se nalaze linije u slede\'c{}em formatu:\\
  \texttt{[<DATUM\_VREME>][<IP\_ADRESA>][<URL\_DO\_RESURSA\_NA\_SERVERU>]} npr.\\
  \texttt{[12.12.2010 17:41:00][123.123.123.123][\url{http://test.com/path/secret.txt}]}\\
  \texttt{[12.12.2010 17:51:03][2607:f0d0:1002:51::4][\url{https://test.com/path/secret.txt}]}\\
  \texttt{[15.12.2010 17:52:11][123.23.2.33][\url{ftp://test.com/secret.txt}]}\\
  \texttt{[15.12.2010 17:52:21][123.23.2.33][\url{FILE:///home/test/secret.txt}]}\\
  Filtrirati pro\v{c}itani sadr\v{z}aj tako da se na standardni izlaz (umesto ispisivanja \v{c}ivavog fajla) ispi\v{s}u samo one linije u kojima je zahtev za resurs stigao preko HTTP ili HTTPS protokola. \hfill (3p)
  \item Format ispisa linije na standardni izlaz promeniti na:\\
  \texttt{v<VERZIJA\_IP\_ADRESE>:<KORI\v{S}\'C{}ENI\_PROTOKOL>:<PUTANJA\_DO\_RESURSA>}\\
  (zameniti tagove odgovaraju\'c{}im informacijama) npr.\\
  \texttt{v4:http:/path/secret.txt} \hfill (2p)
  \item Ispisati samo one linije u kojima je vrednost niske koja predstavlja datum i vreme hronolo\v{s}ki pre trenutnog datuma i vremena. \hfill (4p)
  \item Presko\v{c}iti linije u kojima je unutar URL-a naveden nepodrazumevani port za odgovaraju\'c{}i protokol. \hfill (2p)
  \item Postarati se da u slu\v{c}aju izuzetka aplikacija ispravno zatvori kori\v{s}\'c{}ene resurse i tokove podataka. \hfill (1p)
\end{itemize}
% ----------- END 1 -----------

\vspace{15pt}
\begin{center}
  \textbf{------------------------------------------------------------------------------------------------------------------------------}
\end{center}
\textit{Napomena: Ohrabrujemo studente da koriste \texttt{netcat} kako bi testirali delimi\v{c}ne implementacije i otkrili gre\v{s}ke pre vremena. Takodje, ukoliko se npr. presko\v{c}i implementacija servera, mo\v{z}e se mock-ovati server putem \texttt{netcat}-a.} 
\begin{center}
  \textbf{--------------------------------------------------- Okrenite stranu! ---------------------------------------------------}
\end{center}

\newpage

% ----------- BEGIN 2 -----------
\item \textbf{UDP Sockets (20p/12p)}
\\Napraviti osnovu za klijent-server Java aplikaciju koriste\'c{}i \emph{Java UDP API} koja vr\v{s}i odgovaraju\'c{}e kodiranje niske. 

\begin{itemize}
  \item Napraviti Java aplikaciju koja ima ulogu UDP klijenta. Poslati datagram lokalnom serveru na portu $12345$ koji sadr\v{z}i nisku (koja može da sadrži beline). Niska se učitava sa standardnog ulaza na klijentskoj strani. Primiti datagram koji predstavlja odgovor servera i rezultat ispisati na standardni izlaz. \hfill(8p/4p)
  \item Napraviti Java aplikaciju koja ima ulogu servera koji sluša na portu $12345$. Kada dobije nisku od klijenta, tu nisku transformiše na sledeći način:
  \begin{itemize}
  	\item Ukoliko je karakter veliko slovo, postaje malo slovo i udvostručuje se.
  	\item Ukoliko je karakter malo slovo, postaje veliko slovo.
  	\item Ukoliko je karakter broj, postaje karakter \textbf{'.'} i udvostručuje se.
  \end{itemize} 

  Primer:\\
  niska:  \texttt{mReze Ispit 2}\\
  transformisana niska: \texttt{MrrEZE iiSPIT ..}\\ 
  Poslati datagram po\v{s}iljaocu u kome se nalazi transformisana niska. \hfill(10p/7p)

  \item Postarati se da su svi resursi ispravno zatvoreni u slu\v{c}aju izuzetka. \hfill (2p/1p)
\end{itemize}% ----------- END 2 -----------

\vspace{15pt}

% ----------- BEGIN 3 -----------
\item \textbf{Non-Blocking IO - Loto (25p/18p)}
\\Napraviti klient-server Java aplikaciju koriste\'c{}i \emph{TCP Sockets/Channels API} preko koje se ra\v{c}una broj pogodaka u igri Loto.
\begin{itemize}
    \item Napisati Java klasu koja ima ulogu \textbf{blokiraju\'c{}eg} TCP klijenta koriste\'c{}i Java Channels API. Klijent formira konekciju sa lokanim serverom na portu 12345 i zatim \v{s}alje serveru jednu Loto kombinaciju, to jest 7 celih brojeva iz intervala [1, 39], učitanu sa standardnog ulaza. \hfill (8p/6p)
    \item Napisati Java klasu koja ima ulogu lokalnog \textbf{neblokiraju\'c{}eg} TCP servera, koji oslu\v{s}kuje na portu 12345, koriste\'c{}i Java Channels API. Server slučajnim izborom generiše jednu Loto kombinaciju. Nakon konekcije, klijent \v{s}alje serveru učitanu Loto kombinaciju. Brojevi unutar kombinacije se ne smeju ponavljati. Kao odgovor, server \v{s}alje klijentu informaciju o broju pogodaka u serverski generisanoj kombinaciji. \hfill (15p/11p)
    \item Postarati se da su svi resursi ispravno zatvoreni u slu\v{c}aju izuzetka. \hfill (2p/1p)
  \end{itemize}
% ----------- END 3 -----------

\end{enumerate}

\end{document}
% ----------- END DOCUMENT -----------
