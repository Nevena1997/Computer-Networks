\documentclass[]{article}

% ----------- BEGIN PACKAGES -----------
\usepackage[utf8]{inputenc}
\usepackage[english,serbian]{babel}
\usepackage[margin=0.7in]{geometry}
\usepackage{url}
\usepackage{float}
\usepackage[graphicx]{realboxes}
\usepackage{listings}
\usepackage{textcomp}
\usepackage{xcolor}
\usepackage{titlesec}
\usepackage{adjustbox}
\lstset {
    language=Java,
    frame=none,
    % xleftmargin=-.25in,
    % xrightmargin=.25in
    framesep=10pt,
    tabsize=4,
    showstringspaces=false,
    upquote=true,
    commentstyle=\color{black},
    keywordstyle=\color{black},
    stringstyle=\color{black},
    basicstyle=\small\ttfamily,
    emph={int,char,double,float,unsigned,void,bool},
    emphstyle={\color{black}},
    escapechar=\&,
    classoffset=1,
    morekeywords={>,<,.,;,,,-,!,=,~},
    keywordstyle=\color{black},
    classoffset=0,
    breaklines=true
}
\setlength{\tabcolsep}{30pt}
\pagenumbering{gobble}
% ----------- END PACKAGES -----------

% ----------- BEGIN PREAMBLE -----------
\titlespacing\title{left spacing}{before spacing}{after spacing}[right]

\title{Ra\v{c}unarske mre\v{z}e 4I, Ispit - SEP1}
\date{26.08.2020.}

\begin{document}

\makeatletter
\begin{center}

{\fontsize{12pt}{14pt}\selectfont\bfseries\@title\par}
\@date
\vspace{5mm}

\noindent\fbox{%
    \parbox{\textwidth}{%
      Pro\v{c}itati sve zadatke \textbf{pa\v{z}ljivo} pre rada - sve \v{s}to nije navedeno ne mora da se implementira! 

      Na \texttt{Desktop}-u se nalazi zip arhiva. Unutar arhive se nalazi direktorijum u formatu \texttt{rm\_rok\_Ime\_Prezime\_miGGXXX}\\
      u kome se nalazi validan IntelliJ projekat. Izvu\'c{}i direktorijum iz arhive na Desktop i ubaciti svoje podatke u ime.\\
      Otvoriti IntelliJ IDEA, izabrati opciju \texttt{Open project} (ne \texttt{Import project}!) i otvoriti pomenuti direktorijum.\\ 
      Sve kodove ostaviti unutar ve\'c{} kreiranih Java fajlova. \textbf{Kodovi koji se ne prevode se ne\'c{}e pregledati.}\\
      \textbf{Nepo\v{s}tovanje formata ulaza/izlaza nosi kaznu od -10\% poena na zadatku!}
      Vreme za rad: \textbf{2.5h}.
    }%
}
\end{center}
\makeatother
% ----------- END PREAMBLE -----------

\vspace{5pt}

% ----------- BEGIN DOCUMENT -----------
\begin{enumerate}
  
% ----------- BEGIN 1 -----------
\item \textbf{Non-Blocking IO (18p)}
\\Za potrebe skeniranja terena odlu\v{c}eno je da se kreira \emph{hab} (server) koji \'c{}e prikupljati podatke koji se dobijaju od \emph{skenera} (klijenata). Svaki skener se odlikuje svojom pozicijom $(x,y)$ i mo\v{z}e da pokrije odgovaraju\'c{}u oblast radijusa $r$. Cilj je pokriti \v{c}itav teren skenerima. Teren se posmatra kao mre\v{z}a jedini\v{c}nih kvadrata du\v{z}ine $m$ i \v{s}irine $n$ (videti sliku ispod), dok je $(x,y)$ jedno teme mre\v{z}e. Skener pokriva kvadrat veli\v{c}ine $2r$ sa centrom u $(x,y)$. $m$, $n$, $x$, $y$ i $r$ su nenegativni celi brojevi.

\begin{figure}[h!]
  \centering
  \includegraphics[scale=0.6]{images/terrain.png}
  \caption{Teren dimenzija $9 \times 6$ sa tri postavljena skenera. Pokrivenost u ovom slu\v{c}aju: $\frac{22}{9 \times 6} \approx 0.407 = 40.7\%$}
\end{figure}

\begin{itemize}
  \item Napraviti Java aplikaciju koja ima ulogu skenera. Povezati se na lokalni server na portu $7337$ koristeći \textbf{blokiraju\'c{}i Java Channels API}. Nakon formiranja konekcije, klijent serveru \v{s}alje brojeve $(x,y)$ i $r$ koje operater skenera unosi sa standardnog ulaza. Nakon slanja, klijent ispisuje odgovore od servera sve dok server ne prekine vezu. \hfill (4p)
  \item Napraviti Java aplikaciju koja ima ulogu centralnog haba. Pokrenuti lokalni server na portu $7337$, koriste\'c{}i \textbf{neblokiraju\'c{}i Java Channels API}. Prilikom pokretanja, operater haba unosi veli\v{c}inu terena --- brojeve $m$ i $n$. Hab zatim opslu\v{z}uje skenere. Nakon uspostavljanja veze, hab prima podatke od skenera --- $(x,y)$ i $r$ (pozicija skenera mora biti unutar granica terena, u protivnom se raskida veza sa tim skenerom). U međuvremenu, na svakih $5$ sekundi, hab svim skenerima \v{s}alje trenutnu pokrivenost terena --- procenat jedini\v{c}nih kvadrata koji su pokriveni skenerima u odnosu na ukupan broj jedini\v{c}nih kvadrata u mre\v{z}i. \hfill (9p)
  \item Server raskida vezu sa klijentima i zavr\v{s}ava sa radom onda kada pokrivenost terena postane $100\%$. \hfill (2p)
  \item Obezbediti da u slučaju izuzetaka, svi resursi budu ispravno zatvoreni i da se ukupna pokrivenost terena eventualno promeni ukoliko je neki skener prekinuo vezu! \hfill (3p)
\end{itemize}
% ----------- END 1 -----------

\vspace{5pt}
\begin{center}
  \textbf{------------------------------------------------------------------------------------------------------------------------------}
\end{center}
\textit{Napomena: Ohrabrujemo studente da koriste \texttt{netcat} kako bi testirali delimi\v{c}ne implementacije i otkrili gre\v{s}ke pre vremena. Takodje, ukoliko se npr. presko\v{c}i implementacija servera, mo\v{z}e se mock-ovati server putem \texttt{netcat}-a.} 
\begin{center}
  \textbf{--------------------------------------------------- Okrenite stranu! ---------------------------------------------------}
\end{center}
\newpage

% ----------- BEGIN 2 -----------
\item \textbf{UDP Sockets (12p)}
\\Implementirati UDP server koji na osnovu informacija o terenu i postavljenim skenerima klijentima \v{s}alje odgovor o pokrivenosti njihove lokacije (temena mre\v{z}e terena). 
\begin{itemize}
  \item Napraviti Java aplikaciju koja ima ulogu UDP klijenta. Poslati inicijalni datagram lokalnom serveru na portu $12345$ sa lokacijom klijenta na mre\v{z}i terena (videti sliku sa prethodne strane). Ispisati \texttt{Pokriven!} ako je server odgovorio pozitivno na datagram (lokacija je pokrivena bar jednim skenerom) ili \texttt{Nije pokriven!} ukoliko je server odgovorio negativno na datagram (informacije proizvoljno kodirati u datagrame). \hfill (3p)
  \item Napraviti Java aplikaciju koja ima ulogu UDP servera. Najpre iz fajla \texttt{terrain.txt} u\v{c}itati podatke o terenu i postavljenim skenerima. Primer fajla je dat ispod. \hfill (2p)
  \item Nakon u\v{c}itavanja podataka o terenu, slu\v{s}ati na portu $12345$ i ispisati tekst \texttt{Pristigao klijent!} kad god server primi datagram. Odgovoriti datagramom koji sadr\v{z}i informaciju o pokrivenosti lokacije izvu\v{c}ene iz datagrama --- lokacija je pokrivena ako je u radijusu barem jednog skenera (informacije proizvoljno kodirati u datagrame). \hfill (6p)
  \item Postarati se da su svi resursi ispravno zatvoreni u slu\v{c}aju izuzetka. \hfill (1p)
\end{itemize}

\vspace{10pt}

\begin{figure}[h!]
  \noindent
  \begin{lstlisting}
    9 6
    3 4 1
    2 0 1
    6 2 2
  \end{lstlisting}
  \caption{Primer fajla \texttt{terrain.txt} koji odgovara slici sa prethodne strane --- u prvoj liniji se nalaze dimenzije terena, a zatim u narednim linijama se nalaze informacije o postavljenim skenerima, po jedan skener u svakoj liniji (redosled nije bitan, u ovom primeru je redosled: plavi, zeleni pa crveni).}
\end{figure}

\vspace{15pt}
Primer izvr\v{s}avanja:

\vspace{10pt}
\noindent
\begin{tabular}{ll}
\begin{lstlisting}
terrain.txt ucitan!
\end{lstlisting}& \\
\begin{lstlisting}
Server pokrenut!
\end{lstlisting}& \\
&\begin{lstlisting}
Klijent pokrenut!
\end{lstlisting}\\
&\begin{lstlisting}
> 2 1
\end{lstlisting}\\
\begin{lstlisting}
Stigao datagram!
\end{lstlisting}& \\
\begin{lstlisting}
\end{lstlisting}&\begin{lstlisting}
Pokriven!
\end{lstlisting}\\
&\begin{lstlisting}
Klijent pokrenut!
\end{lstlisting}\\
&\begin{lstlisting}
> 2 2
\end{lstlisting}\\
\begin{lstlisting}
Stigao datagram!
\end{lstlisting}& \\
\begin{lstlisting}
\end{lstlisting}&\begin{lstlisting}
Nije Pokriven!
\end{lstlisting}\\
&\begin{lstlisting}
Klijent pokrenut!
\end{lstlisting}\\
&\begin{lstlisting}
> 7 3
\end{lstlisting}\\
\begin{lstlisting}
Stigao datagram!
\end{lstlisting}& \\
\begin{lstlisting}
\end{lstlisting}&\begin{lstlisting}
Pokriven!
\end{lstlisting}\\
&\begin{lstlisting}
Klijent pokrenut!
\end{lstlisting}\\
&\begin{lstlisting}
> 100 100
\end{lstlisting}\\
\begin{lstlisting}
Stigao datagram!
\end{lstlisting}& \\
\begin{lstlisting}
\end{lstlisting}&\begin{lstlisting}
Nije Pokriven!
\end{lstlisting}\\
\end{tabular}
% ----------- END 2 -----------
\end{enumerate}
\end{document}
% ----------- END DOCUMENT -----------
