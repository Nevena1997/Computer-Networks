\documentclass[]{article}

% ----------- BEGIN PACKAGES -----------
\usepackage[utf8]{inputenc}
\usepackage[english,serbian]{babel}
\usepackage[margin=0.7in]{geometry}
\usepackage{url}
\usepackage{float}
\usepackage[graphicx]{realboxes}
\usepackage{listings}
\usepackage{textcomp}
\usepackage{xcolor}
\usepackage{titlesec}
\usepackage{adjustbox}
\lstset {
    language=Java,
    frame=none,
    % xleftmargin=-.25in,
    % xrightmargin=.25in
    framesep=10pt,
    tabsize=4,
    showstringspaces=false,
    upquote=true,
    commentstyle=\color{black},
    keywordstyle=\color{black},
    stringstyle=\color{black},
    basicstyle=\small\ttfamily,
    emph={int,char,double,float,unsigned,void,bool},
    emphstyle={\color{black}},
    escapechar=\&,
    classoffset=1,
    morekeywords={>,<,.,;,,,-,!,=,~},
    keywordstyle=\color{black},
    classoffset=0,
    breaklines=true
}
\setlength{\tabcolsep}{30pt}
\pagenumbering{gobble}
% ----------- END PACKAGES -----------

% ----------- BEGIN PREAMBLE -----------
\titlespacing\title{left spacing}{before spacing}{after spacing}[right]

\title{Ra\v{c}unarske mre\v{z}e 4I, Ispit - JUN1}
\date{17.06.2020.}

\begin{document}

\makeatletter
\begin{center}

{\fontsize{12pt}{14pt}\selectfont\bfseries\@title\par}
\@date
\vspace{5mm}

\noindent\fbox{%
    \parbox{\textwidth}{%
      Pro\v{c}itati sve zadatke \textbf{pa\v{z}ljivo} pre rada - sve \v{s}to nije navedeno ne mora da se implementira! 

      Na \texttt{Desktop}-u se nalazi zip arhiva. Unutar arhive se nalazi direktorijum u formatu \texttt{rm\_rok\_Ime\_Prezime\_miGGXXX}\\
      u kome se nalazi validan IntelliJ projekat. Izvu\'c{}i direktorijum iz arhive na Desktop i ubaciti svoje podatke u ime.\\
      Otvoriti IntelliJ IDEA, izabrati opciju \texttt{Open project} (ne \texttt{Import project}!) i otvoriti pomenuti direktorijum.\\ 
      Sve kodove ostaviti unutar ve\'c{} kreiranih Java fajlova. \textbf{Kodovi koji se ne prevode se ne\'c{}e pregledati.}\\
      \textbf{Nepo\v{s}tovanje formata ulaza/izlaza nosi kaznu od -10\% poena na zadatku!}
      Vreme za rad: \textbf{2.5h}.
    }%
}
\end{center}
\makeatother
% ----------- END PREAMBLE -----------

\vspace{5pt}

% ----------- BEGIN DOCUMENT -----------
\begin{enumerate}
  
% ----------- BEGIN 1 -----------
\item \textbf{TCP Sockets (16p)}
\\Implementirati server koji imitira anonimni online forum sa spiskom \emph{tema} (svaka tema mo\v{z}e imati proizvoljno mnogo odgovora u okviru nje). Korisnici foruma imaju mogućnost da postavljaju nove teme ili da odgovore na već postojeće teme.
\begin{itemize}
  \item Napraviti Java aplikaciju koja ima ulogu klijenta. Povezati se na lokalni server na portu $7337$ koristeći \textbf{Socket} klasu. Nakon formiranja konekcije, klijent mo\v{z}e poslati vi\v{s}e zahteva serveru (zahtevi se unose sa standardnog ulaza), sve dok mu ne pošalje \texttt{bye}. Odgovori servera na zahtev se ispisuju na standardni izlaz. Mogu\'c{}i zahtevi su (implementirati u ovoj stavci samo slanje od strane klijenta, uz provere da li je unos validan): \hfill (2p)
  \begin{itemize}
      \item \textit{list}
      \item \textit{reply} \texttt{id content} (\texttt{id} je tipa \texttt{int}, \texttt{content} je tipa \texttt{String}) 
      \item \textit{post} \texttt{title content} (\texttt{title} je tipa \texttt{String}, \texttt{content} je tipa \texttt{String})
  \end{itemize}
  \item Napraviti Java aplikaciju koja ima ulogu servera. Pokrenuti lokalni server na portu $7337$, koriste\'c{}i \textbf{ServerSocket} klasu. Server za svakog primljenog klijenta pokre\'c{}e zasebnu nit u kojoj \'c{}e se taj klijent obraditi tako \v{s}to se ispi\v{s}e poruka o pristiglom klijentu kao u primeru ispod. \hfill (2p)
  \item Server pristigle zahteve obradjuje na slede\'c{}i na\v{c}in:
  \begin{itemize}
    \item \textit{list}: vra\'c{}a spisak svih trenutnih registrovanih tema \hfill (2p)
    \item \textit{reply} \texttt{id content}: dodaje se odgovor sa sadr\v{z}ajem \texttt{content} na temu sa identifikatorom \texttt{id} \hfill (3p)
    \item \textit{post} \texttt{title content}: dodaje se nova tema sa naslovom \texttt{title} i sadr\v{z}ajem \texttt{content} (identifikator teme je proizvoljan ali mora biti jedinstven za svaku temu) \hfill (3p)
  \end{itemize}
  \item Ukoliko bilo koji od ovih zahteva nije ispravno formiran ili nije naveden iznad, vratiti poruku o gre\v{s}ci kao u primerima ispod. \hfill(1p)
  \item Imajte u vidu da mogu da se dese potencijalne konfliktne situacije (kao npr. da dva klijenta žele da odgovore u isto vreme na istu temu). Obezbediti da se ovakvi odgovori pravilno obrade i da ne postoje konkurentni problemi. Takođe, obezbediti da u slučaju izuzetaka, svi resursi budu ispravno zatvoreni. \hfill (3p)
\end{itemize}

\vspace{15pt}
\begin{center}
  \textbf{------------------------------------------------------------------------------------------------------------------------------}
\end{center}
\textit{Napomena: Ohrabrujemo studente da koriste \texttt{netcat} kako bi testirali delimi\v{c}ne implementacije i otkrili gre\v{s}ke pre vremena. Takodje, ukoliko se npr. presko\v{c}i implementacija servera, mo\v{z}e se mock-ovati server putem \texttt{netcat}-a.} 
\begin{center}
  \textbf{--------------------------------------------------- Okrenite stranu! ---------------------------------------------------}
\end{center}
\newpage

\noindent
\begin{tabular}{ll}
\begin{lstlisting}
> post "Zdravo svete!" "Prvi post!"
post je uspesno izvrsen
> post "IDEA" "Precica za run?"
post je uspesno izvrsen
> reply 1 "It works!"
reply je uspesno izvrsen
> reply 1 "yay!"
reply je uspesno izvrsen
> reply 2 "Shift+F10"
reply je uspesno izvrsen
> list
1: Zdravo svete
  # Prvi post!
  - It works!
  - yay!
2: IDEA
  # Precica za run?
  - Shift+F10
> bye
\end{lstlisting}&
\begin{lstlisting}
> post "Prvi post!"
nevalidan format
> reply "test"
nevalidan format
> reply 1 2 "It works!"
nevalidan format
> reply 100 "test"
nevalidan id teme
> list
1: Zdravo svete
  # Prvi post!
  - It works!
  - yay!
2: IDEA
  # Precica za run?
  - Shift+F10
> bye
\end{lstlisting}
\end{tabular}
% ----------- END 1 -----------

\vspace{15pt}

% ----------- BEGIN 2 -----------
\item \textbf{Implementacija forum protokol handler-a (14p)}
\\Implementirati podr\v{s}ku za URL-ove koji koriste \texttt{forum} protokol. Opis protokola je dat u prvom zadatku. 
\begin{itemize}
  \item Prilikom otvaranja konekcije, formirati vezu koriste\'c{}i Socket API. Povezati se na server i port na osnovu URL-a i otvoriti ulazni tok do odgovora na upit \texttt{list} od strane servera. \hfill (5p)
  \item Omogu\'c{}iti slanje upita pomo\'c{}u parametra \texttt{q} iz URL-a, npr za upit \texttt{post}, kompletan URL bi bio:
  \begin{lstlisting}
    forum://localhost:7337?q=post&title=Zdravo&content=Paziti+na+specijalne+karaktere
  \end{lstlisting}
  Server \v{s}alje potencijalno a\v{z}urirani spisak tema i odgovora. \hfill (5p)
  \item Ukoliko port nije naveden unutar URL-a, koristiti podrazumevani port isti kao u prvom zadatku. \hfill (1p)
  \item Predefinisati \texttt{getInputStream()} metod da vra\'c{}a ulazni tok do odgovora od strane servera ukoliko je konekcija ostvarena, a \texttt{null} ako nije. \hfill (1p)
  \item Postarati se da je mogu\'c{}e bezbedno koristiti implementirani handler u vi\v{s}enitnom okru\v{z}enju. \hfill (1p)
  \item Napisati jednostavan test - kreirati URL, otvoriti konekciju do resursa i ispisati preuzeti sadr\v{z}aj (videti primere u nastavku). \hfill (1p)
\end{itemize}

\vspace{10pt}
\noindent
  \begin{lstlisting}
    URL:   forum://localhost
    izlaz: 1: Zdravo svete
             # Prvi post!
             - It works!
             - yay!
           2: IDEA
             # Precica za run?
             - Shift+F10
  \end{lstlisting}
  \begin{lstlisting}
    URL:   forum://localhost:7337?q=reply&id=1&content=test
    izlaz: 1: Zdravo svete
             # Prvi post!
             - It works!
             - yay!
             - test
           2: IDEA
             # Precica za run?
             - Shift+F10
  \end{lstlisting}
  \begin{lstlisting}
    URL:   forum://localhost:12345
    izlaz: Neuspela konekcija.
  \end{lstlisting}
  \begin{lstlisting}
    URL:   forum://localhost?q=post
    izlaz: nevalidan format
  \end{lstlisting}
% ----------- END 2 -----------
\end{enumerate}
\end{document}
% ----------- END DOCUMENT -----------
