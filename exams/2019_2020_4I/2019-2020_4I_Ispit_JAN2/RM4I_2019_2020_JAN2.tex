\documentclass[]{article}

% ----------- BEGIN PACKAGES -----------
\usepackage[utf8]{inputenc}
\usepackage[english,serbian]{babel}
\usepackage[margin=0.7in]{geometry}
\usepackage{url}
\usepackage{float}
\usepackage[graphicx]{realboxes}
\usepackage{listings}
\usepackage{textcomp}
\usepackage{xcolor}
\usepackage{titlesec}
\usepackage{adjustbox}
\lstset {
    language=Java,
    frame=none,
    %xleftmargin=-.25in,
    %xrightmargin=.25in
    framesep=10pt,
    tabsize=4,
    showstringspaces=false,
    upquote=true,
    commentstyle=\color{black},
    keywordstyle=\color{black},
    stringstyle=\color{black},
    basicstyle=\small\ttfamily,
    emph={int,char,double,float,unsigned,void,bool},
    emphstyle={\color{black}},
    escapechar=\&,
    classoffset=1,
    morekeywords={>,<,.,;,,,-,!,=,~},
    keywordstyle=\color{black},
    classoffset=0,
    breaklines=true
}
\pagenumbering{gobble}
% ----------- END PACKAGES -----------

% ----------- BEGIN PREAMBLE -----------
\titlespacing\title{left spacing}{before spacing}{after spacing}[right]

\title{Ra\v{c}unarske mre\v{z}e 4I, Ispit - JAN2}
\date{24.01.2020.}

\begin{document}

\makeatletter
\begin{center}

{\fontsize{12pt}{14pt}\selectfont\bfseries\@title\par}
\@date
\vspace{5mm}

\noindent\fbox{%
    \parbox{\textwidth}{%
      Pro\v{c}itati sve zadatke \textbf{pa\v{z}ljivo} pre rada - sve \v{s}to nije navedeno ne mora da se implementira! 

      Na \texttt{Desktop}-u se nalazi zip arhiva. Unutar arhive se nalazi direktorijum sa imenom \texttt{rm\_rok\_Ime\_Prezime\_miGGXXX}\\
      u kome se nalazi validan IntelliJ projekat. Izvu\'c{}i direktorijum iz arhive na Desktop i zameniti svojim podacima.\\
      Otvoriti IntelliJ IDEA, izabrati opciju \texttt{Open project} (ne \texttt{Import project}!) i otvoriti pomenuti direktorijum.\\ 
      Sve kodove ostaviti unutar ve\'c{} kreiranih Java fajlova. \textbf{Kodovi koji se ne prevode se ne\'c{}e pregledati.}\\
      \textbf{Nepo\v{s}tovanje formata ulaza/izlaza nosi kaznu od -10\% poena na zadatku!}
      Vreme za rad: \textbf{2.5h}.
    }%
}
\end{center}
\makeatother
% ----------- END PREAMBLE -----------

\vspace{5pt}

% ----------- BEGIN DOCUMENT -----------
\begin{enumerate}

% ----------- BEGIN 1 -----------
  \item \textbf{Daytime klijent-server (20p)}
  \\Implementirati serverski program koji povezanim klijentima na svakih $5$ sekundi \v{s}alje trenutno vreme u formatu:\\
  \texttt{dd.mm.yyyy | hh:mm:ss} (videti primer poziva ispod).
  \begin{itemize}
    \item Napraviti Java aplikaciju koja ima ulogu servera. Pokrenuti \textbf{neblokiraju\'c{}i} lokalni server na portu 12345 koriste\'c{}i Java Channels API. Kada do servera pristigne klijent, server ga prihvata i zapo\v{c}inje obradu. Inicijalno implementirati da server klijentima \v{s}alje bilo kakvu nisku koja sadr\v{z}i trenutno vreme, samo jednom. \hfill (3p)
    \item Postarati se da server klijentima periodi\v{c}no svakih $5$ sekundi \v{s}alje trenutno vreme. Dakle, svaki klijent dobija novu nisku nakon $5$ sekundi. Takodje, server ne raskida vezu do klijenta nakon jednog slanja ve\'c{} mu \v{s}alje poruke sve dok klijent ne prekine vezu. \hfill (4p)
    \item Postarati se da je niska koju server \v{s}alje klijentima u formatu:\\
    \texttt{dd.mm.yyyy | hh:mm:ss} \hfill (2p)
    \item Postarati se da server nastavi sa radom ukoliko neki klijent raskine vezu. \hfill (2p)
    \item Napraviti Java aplikaciju koja ima ulogu klijenta. Povezati se na lokalni server na portu 12345 koriste\'c{}i \textbf{blokiraju\'c{}i} Java Channels API i \v{c}itati podatke primljene od servera u beskona\v{c}noj petlji. Primljene podatke ispisivati na standardni izlaz bez izmena. \hfill (2p)
    \item Izmeniti klijenta tako da se primanje podataka od servera vr\v{s}i u zasebnoj niti, dok program u glavnoj niti \v{c}eka na unos sa standardnog ulaza koji predstavlja ``signal'' za prekid rada klijentskog programa. Kada korisnik zatra\v{z}i zaustavljanje klijenta, klijent se zaustavlja. \hfill (6p)
    \item Postarati se da su svi resursi ispravno zatvoreni u slu\v{c}aju izuzetka (za oba programa). \hfill (1p)
  \end{itemize}

  \vspace{10pt}
  \noindent
  \begin{tabular}{lll}
  \begin{lstlisting}
  Klijent 1:
  \end{lstlisting}&
  \begin{lstlisting}
  Klijent 2:
  \end{lstlisting}&
  \begin{lstlisting}
  Server:
  \end{lstlisting}\\
  \begin{lstlisting}
  \end{lstlisting}&
  \begin{lstlisting}      
  \end{lstlisting}&
  \begin{lstlisting}
  [start]
  \end{lstlisting}\\
  \begin{lstlisting}
  [start]
  \end{lstlisting}&
  \begin{lstlisting}      
  \end{lstlisting}&
  \begin{lstlisting}
  \end{lstlisting}\\
  \begin{lstlisting}
  Connecting...
  \end{lstlisting}&
  \begin{lstlisting}      
  \end{lstlisting}&
  \begin{lstlisting}
  Accepted client
  \end{lstlisting}\\
  \begin{lstlisting}
  \end{lstlisting}&
  \begin{lstlisting}    
  \end{lstlisting}&
  \begin{lstlisting}
  Serving time...
  \end{lstlisting}\\
  \begin{lstlisting}
  24.01.2020 | 09:00:00
  \end{lstlisting}&
  \begin{lstlisting}  
  [start]      
  \end{lstlisting}&
  \begin{lstlisting}
  \end{lstlisting}\\
  \begin{lstlisting}
  \end{lstlisting}&
  \begin{lstlisting} 
  Connecting...
  \end{lstlisting}&
  \begin{lstlisting}
  Accepted client
  \end{lstlisting}\\
  \begin{lstlisting}
  \end{lstlisting}&
  \begin{lstlisting} 
  \end{lstlisting}&
  \begin{lstlisting}
  Serving time...
  \end{lstlisting}\\
  \begin{lstlisting}
  24.01.2020 | 09:00:05
  \end{lstlisting}&
  \begin{lstlisting}   
  24.01.2020 | 09:00:05
  \end{lstlisting}&
  \begin{lstlisting}
  \end{lstlisting}\\
  \begin{lstlisting}
  [end]
  \end{lstlisting}&
  \begin{lstlisting}   
  \end{lstlisting}&
  \begin{lstlisting}
  Client disconnected
  \end{lstlisting}\\
  \begin{lstlisting}
  \end{lstlisting}&
  \begin{lstlisting}   
  \end{lstlisting}&
  \begin{lstlisting}
  Serving time...
  \end{lstlisting}\\
  \begin{lstlisting}
  \end{lstlisting}&
  \begin{lstlisting}   
  24.01.2020 | 09:00:10
  \end{lstlisting}&
  \begin{lstlisting}
  \end{lstlisting}\\
  \begin{lstlisting}
  \end{lstlisting}&
  \begin{lstlisting}   
  [end]
  \end{lstlisting}&
  \begin{lstlisting}
  Client disconnected
  \end{lstlisting}
  \end{tabular}
% ----------- END 1 -----------
\begin{center}
  \vspace{20pt}
  \textbf{------------------------------------------------------------------------------------------------------------------------------}
\end{center}
\vspace{10pt}
\textit{Napomena: Ohrabrujemo studente da koriste \texttt{netcat} kako bi testirali delimi\v{c}ne implementacije i otkrili gre\v{s}ke pre vremena. Takodje, ukoliko se npr. presko\v{c}i implementacija servera, mo\v{z}e se mock-ovati server putem \texttt{netcat}-a.} 
\begin{center}
  \vspace{10pt}
  \textbf{--------------------------------------------------- Okrenite stranu! ---------------------------------------------------}
\end{center}
\newpage
% ----------- BEGIN 2 -----------
  \item \textbf{Implementacija daytime protokol handler-a (10p)}
  \\Implementirati podr\v{s}ku za URL-ove koji koriste \texttt{daytime} protokol. Opis protokola je dat u prvom zadatku. 
  \begin{itemize}
    \item Prilikom otvaranja konekcije, formirati vezu koriste\'c{}i Socket API. Povezati se na server i port na osnovu URL-a i otvoriti ulazni tok do odgovora od strane servera. \hfill (5p)
    \item Ukoliko port nije naveden unutar URL-a, koristiti podrazumevani port isti kao u prvom zadatku. \hfill (1p)
    \item Predefinisati \texttt{getInputStream()} metod da vra\'c{}a ulazni tok do odgovora od strane servera ukoliko je konekcija ostvarena, a \texttt{null} ako nije. \hfill (1p)
    \item Postarati se da je mogu\'c{}e bezbedno koristiti implementirani handler u vi\v{s}enitnom okru\v{z}enju. \hfill (1p)
    \item Napisati jednostavan test - kreirati URL, otvoriti konekciju do resursa i preuzeti sadr\v{z}aj (videti primere u nastavku). \hfill (2p)
  \end{itemize}
  
  \vspace{10pt}
  \noindent
    \begin{lstlisting}
      URL:   daytime://localhost:12345
      izlaz: 24.01.2020 | 09:00:00
             24.01.2020 | 09:00:05
             24.01.2020 | 09:00:10
    \end{lstlisting}
    \begin{lstlisting}
      URL:   daytime://localhost
      izlaz: 24.01.2020 | 09:00:00
             24.01.2020 | 09:00:05
             24.01.2020 | 09:00:10
    \end{lstlisting}
    \begin{lstlisting}
      URL:   daytime://nonexistingserver:12345
      izlaz: Connection failed.
    \end{lstlisting}
    \begin{lstlisting}
      URL:   daytime://nonexistingserver:9999
      izlaz: Connection failed.
    \end{lstlisting}
% ----------- END 2 -----------
\end{enumerate}
\end{document}
% ----------- END DOCUMENT -----------
