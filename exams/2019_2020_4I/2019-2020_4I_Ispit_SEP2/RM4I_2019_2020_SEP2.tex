\documentclass[]{article}

% ----------- BEGIN PACKAGES -----------
\usepackage[utf8]{inputenc}
\usepackage[english,serbian]{babel}
\usepackage[margin=0.7in]{geometry}
\usepackage{url}
\usepackage{float}
\usepackage[graphicx]{realboxes}
\usepackage{listings}
\usepackage{textcomp}
\usepackage{xcolor}
\usepackage{titlesec}
\usepackage{adjustbox}
\lstset {
    language=Java,
    frame=none,
    % xleftmargin=-.25in,
    % xrightmargin=.25in
    framesep=10pt,
    tabsize=4,
    showstringspaces=false,
    upquote=true,
    commentstyle=\color{black},
    keywordstyle=\color{black},
    stringstyle=\color{black},
    basicstyle=\small\ttfamily,
    emph={int,char,double,float,unsigned,void,bool},
    emphstyle={\color{black}},
    escapechar=\&,
    classoffset=1,
    morekeywords={>,<,.,;,,,-,!,=,~},
    keywordstyle=\color{black},
    classoffset=0,
    breaklines=true
}
\setlength{\tabcolsep}{30pt}
\pagenumbering{gobble}
% ----------- END PACKAGES -----------

% ----------- BEGIN PREAMBLE -----------
\titlespacing\title{left spacing}{before spacing}{after spacing}[right]

\title{Ra\v{c}unarske mre\v{z}e 4I, Ispit - SEP2}
\date{09.09.2020.}

\begin{document}

\makeatletter
\begin{center}

{\fontsize{12pt}{14pt}\selectfont\bfseries\@title\par}
\@date
\vspace{5mm}

\noindent\fbox{%
    \parbox{\textwidth}{%
      Pro\v{c}itati sve zadatke \textbf{pa\v{z}ljivo} pre rada - sve \v{s}to nije navedeno ne mora da se implementira! 

      Na \texttt{Desktop}-u se nalazi zip arhiva. Unutar arhive se nalazi direktorijum u formatu \texttt{rm\_rok\_Ime\_Prezime\_miGGXXX}\\
      u kome se nalazi validan IntelliJ projekat. Izvu\'c{}i direktorijum iz arhive na Desktop i ubaciti svoje podatke u ime.\\
      Otvoriti IntelliJ IDEA, izabrati opciju \texttt{Open project} (ne \texttt{Import project}!) i otvoriti pomenuti direktorijum.\\ 
      Sve kodove ostaviti unutar ve\'c{} kreiranih Java fajlova. \textbf{Kodovi koji se ne prevode se ne\'c{}e pregledati.}\\
      \textbf{Nepo\v{s}tovanje formata ulaza/izlaza nosi kaznu od -10\% poena na zadatku!}
      Vreme za rad: \textbf{2h}.
    }%
}
\end{center}
\makeatother
% ----------- END PREAMBLE -----------

\vspace{5pt}

% ----------- BEGIN DOCUMENT -----------
\begin{enumerate}
  
% ----------- BEGIN 1 -----------
\item \textbf{Non-Blocking IO (16p)}
\\Napraviti osnovu za klijent-server Java aplikaciju koja izvr\v{s}ava bankovske transakcije. Prilikom povezivanja na server klijenti dobijaju spisak brojeva ra\v{c}una svih trenutno povezanih klijenata. Podrazumevano, svi klijenti su primaoci sve dok ne odaberu broj ra\v{c}una na koji \v{z}ele da po\v{s}alju novac. U tom trenutku klijent postaje po\v{s}iljalac i server o\v{c}ekuje od njega iznos koji \v{z}eli da prebaci na odredi\v{s}ni ra\v{c}un. U ovoj inicijalnoj verziji aplikacije se ne o\v{c}ekuje potpuna implementacija korisni\v{c}kih ra\v{c}una ve\'c{} se podrazumeva da svaki transfer uspeva.

\begin{itemize}
  \item Napraviti Java aplikaciju koja ima ulogu klijenta. Povezati se na lokalni server na portu $12221$ koristeći \textbf{blokiraju\'c{}i Java Channels API}. Nakon formiranja konekcije klijent serveru \v{s}alje svoj broj ra\v{c}una a zatim o\v{c}ekuje od servera spisak brojeva ra\v{c}una svih trenutno povezanih klijenata (po jedan u svakom redu) i ispisuje isti na standardni izlaz. Nakon toga, klijent \v{c}eka na unos broja ra\v{c}una primaoca sa standardnog ulaza (broj ra\v{c}una ne mora da se unese) i \v{s}alje ga serveru zajedno sa iznosom koji predstavlja neozna\v{c}eni ceo broj takođe u\v{c}itan sa standarnog ulaza. Nakon slanja ovih podataka, server odgovara podrvrdom o izvr\v{s}enju transakcije i klijent ispisuje odgovor na standardni izlaz. Primalac takođe prima poruku o izvr\v{s}enju transakcije i istu ispisuje na standardni izlaz. \hfill (5p)
  \item Napraviti Java aplikaciju koja ima ulogu servera. Otvoriti lokalni server na portu $12221$ koristeći \textbf{neblokiraju\'c{}i Java Channels API}. Nakon primanja klijenta, server od njega o\v{c}ekuje broj ra\v{c}una i nakon primanja broja ra\v{c}una mu \v{s}alje brojeve ra\v{c}una svih trenutno povezanih klijenata. Nakon toga, server \v{c}eka na poruke od klijenata. Oni klijenti koji po\v{s}alju broj ra\v{c}una primaoca se tuma\v{c}e kao po\v{s}iljaoci i od njih se dodatno o\v{c}ekuje i iznos koji \v{z}ele da prebace na ra\v{c}un primaoca. Server svakom po\v{s}iljaocu \v{s}alje nisku \texttt{"Transfer na racun BROJ\_RACUNA (IZNOS) je uspesno izvrsen"}, a svakom primaocu \v{s}alje nisku \texttt{"Primljeno IZNOS od BROJ\_RACUNA"} (\textbf{nije neophodno implementirati stanje ra\v{c}una na serverskoj strani, pretpostaviti da svaki transfer uspeva}). \hfill (8p)
  \item Obezbediti da u slučaju izuzetaka, svi resursi budu ispravno zatvoreni i da se po\v{s}iljaocu po\v{s}alje poruka \texttt{Primalac nije pronadjen} ukoliko je primalac prekinuo vezu pre iniciranja transfera. \hfill (3p)
\end{itemize}
% ----------- END 1 -----------

\vspace{20pt}
\begin{center}
  \textbf{------------------------------------------------------------------------------------------------------------------------------}
\end{center}
\textit{Napomena: Ohrabrujemo studente da koriste \texttt{netcat} kako bi testirali delimi\v{c}ne implementacije i otkrili gre\v{s}ke pre vremena. Takodje, ukoliko se npr. presko\v{c}i implementacija servera, mo\v{z}e se mock-ovati server putem \texttt{netcat}-a.} 
\begin{center}
  \textbf{--------------------------------------------------- Okrenite stranu! ---------------------------------------------------}
\end{center}
\newpage

% ----------- BEGIN 2 -----------
\item \textbf{UDP Sockets (14p)}
    \\Implementirati UDP server koji korisnicima daje uvid u stanje njihovog ra\v{c}una. Ra\v{c}uni se mogu u\v{c}itati tokom rada servera koji je du\v{z}an da te ra\v{c}une otvori u toku svog izvr\v{s}avanja. Klijenti slanjem brojem ra\v{c}una serveru putem datagrama mogu dobiti uvid u trenutno stanje ra\v{c}una.
\begin{itemize}
  \item Napraviti Java aplikaciju koja ima ulogu UDP klijenta. Poslati inicijalni datagram lokalnom serveru na portu $12345$ sa brojem ra\v{c}una. Primiti datagram od servera koji sadr\v{z}i iznos koji se nalazi na ra\v{c}unu i ispisati iznos na standardni izlaz (informacije proizvoljno kodirati u datagrame). Iznos posmatrati kao pozitivan realan broj. \hfill (3p)
  \item Napraviti Java aplikaciju koja ima ulogu UDP servera.
    \begin{itemize}
      \item Kreirati nit za unos novih ra\v{c}una sa standardnog ulaza. Tokom rada servera, korisnik mo\v{z}e da unese nove brojeve ra\v{c}una u formatu \texttt{BROJ\_RACUNA IZNOS}. Server je du\v{z}an da taj ra\v{c}un doda u spisak aktivnih ra\v{c}una. Iznos posmatrati kao pozitivan realan broj. Pri pokretanju servera smatrati da ne postoji nijedan registrovan aktivni ra\v{c}un. \hfill (6p)
      \item Kreirati nit za slanje/primanje datagrama. Kad god server primi datagram od klijenta, izvla\v{c}i broj ra\v{c}una i odgovara datagramom koji sadr\v{z}i informaciju o iznosu koji je trenutno dodeljen tom ra\v{c}unu (informacije proizvoljno kodirati u datagrame). Ukoliko broj ra\v{c}una nije aktivan, poslati \texttt{-1}. \hfill (4p)
    \end{itemize}
  \item Postarati se da su svi resursi ispravno zatvoreni u slu\v{c}aju izuzetka. \hfill (1p)
\end{itemize}
% ----------- END 2 -----------

\end{enumerate}
\end{document}
% ----------- END DOCUMENT -----------
