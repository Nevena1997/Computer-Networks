\documentclass[]{article}

% ----------- BEGIN PACKAGES -----------
\usepackage[utf8]{inputenc}
\usepackage[english,serbian]{babel}
\usepackage[margin=0.7in]{geometry}
\usepackage{url}
\usepackage{float}
\usepackage[graphicx]{realboxes}
\usepackage{listings}
\usepackage{textcomp}
\usepackage{xcolor}
\usepackage{titlesec}
\usepackage{adjustbox}
\lstset {
    language=Java,
    frame=none,
    %xleftmargin=-.25in,
    %xrightmargin=.25in
    framesep=10pt,
    tabsize=4,
    showstringspaces=false,
    upquote=true,
    commentstyle=\color{black},
    keywordstyle=\color{black},
    stringstyle=\color{black},
    basicstyle=\small\ttfamily,
    emph={int,char,double,float,unsigned,void,bool},
    emphstyle={\color{black}},
    escapechar=\&,
    classoffset=1,
    morekeywords={>,<,.,;,,,-,!,=,~},
    keywordstyle=\color{black},
    classoffset=0,
    breaklines=true
}
\pagenumbering{gobble}
% ----------- END PACKAGES -----------

% ----------- BEGIN PREAMBLE -----------
\titlespacing\title{left spacing}{before spacing}{after spacing}[right]

\title{Ra\v{c}unarske mre\v{z}e 4I, Ispit - JAN1}
\date{11.01.2020.}

\begin{document}

\makeatletter
\begin{center}

{\fontsize{12pt}{14pt}\selectfont\bfseries\@title\par}
\@date
\vspace{5mm}

\noindent\fbox{%
    \parbox{\textwidth}{%
      Pro\v{c}itati sve zadatke \textbf{pa\v{z}ljivo} pre rada - sve \v{s}to nije navedeno ne mora da se implementira! 

      Na \texttt{Desktop}-u se nalazi zip arhiva. Unutar arhive se nalazi direktorijum sa imenom \texttt{rm\_rok\_Ime\_Prezime\_miGGXXX}\\
      u kome se nalazi validan IntelliJ projekat. Izvu\'c{}i direktorijum iz arhive na Desktop i zameniti svojim podacima.\\
      Otvoriti IntelliJ IDEA, izabrati opciju \texttt{Open project} (ne \texttt{Import project}!) i otvoriti pomenuti direktorijum.\\ 
      Sve kodove ostaviti unutar ve\'c{} kreiranih Java fajlova. \textbf{Kodovi koji se ne prevode se ne\'c{}e pregledati.}\\
      \textbf{Nepo\v{s}tovanje formata ulaza/izlaza nosi kaznu od -10\% poena na zadatku!}
      Vreme za rad: \textbf{2.5h}.
    }%
}
\end{center}
\makeatother
% ----------- END PREAMBLE -----------

\vspace{5pt}

% ----------- BEGIN DOCUMENT -----------
\begin{enumerate}
  
% ----------- BEGIN 1 -----------
\item \textbf{TCP Sockets (15p)}
\\Implementirati server, koji će imati ulogu da održava \emph{in-memory} tabelu šahista i njihove trenutne rejtinge. Tabela ima kolone: \texttt{id (int)}, \texttt{naziv (String)} i \texttt{elo (int)}.
\begin{itemize}
  \item Napraviti Java aplikaciju koja ima ulogu klijenta. Povezati se na lokalni server na portu $1996$ koristeći \textbf{Socket} klasu. Nakon formiranja konekcije, klijent mo\v{z}e poslati vi\v{s}e zahteva serveru (zahtevi se unose sa standardnog ulaza), sve dok mu ne pošalje \texttt{bye}. Odgovori servera na zahtev se ispisuju na standardni izlaz. Mogući zahtevi su (implementirati u ovoj stavci samo slanje od strane klijenta): \hfill (3p)
  \begin{itemize}
      \item \textit{sel id} (\texttt{id} je tipa \texttt{int})
      \item \textit{ins naziv} (\texttt{naziv} je tipa \texttt{String}) 
      \item \textit{upd id elo} (\texttt{id} i \texttt{elo} su tipa \texttt{int})
  \end{itemize}
  \item Napraviti Java aplikaciju koja ima ulogu servera. Pokrenuti lokalni server na portu $1996$, koristeći \textbf{ServerSocket} klasu. Server za svakog primljenog klijenta pokreće zasebnu nit u kojoj \'c{}e se taj klijent obraditi tako \v{s}to se ispi\v{s}e poruka o pristiglom klijentu kao u primeru ispod. \hfill (2p)
  \item Server pristigle zahteve obradjuje na slede\'c{}i na\v{c}in:
  \begin{itemize}
    \item \textit{sel id}: vra\'c{}a naziv i elo \v{s}ahiste sa datim identifikatorom \texttt{id} \hfill (2p)
    \item \textit{ins naziv}: ubacuje u tabelu \v{s}ahistu sa datim imenom dodeljuju\'c{}i mu jedinstveni identifikator (slede\'c{}i slobodan ceo broj) i elo u vrednosti $1300$ (to je najmanja mogu\'c{}a vrednost za elo) i vra\'c{}a poruku o uspe\v{s}nosti operacije \hfill (2p)
    \item \textit{upd id deltae}: vr\v{s}i izmenu elo vrednosti \v{s}ahiste sa identifikatorom \texttt{id} za \texttt{deltae} i vra\'c{}a poruku o uspe\v{s}nosti operacije \hfill (2p)
  \end{itemize}
  \item Ukoliko bilo koji od ovih zahteva nije ispravno formiran ili nije naveden iznad, vratiti tekst kao u primerima ispod. \hfill(1p)
  \item Imajte u vidu da mogu da se dese konfliktne situacije (kao npr. da dva klijenta žele da promene elo istoj osobi). Obezbediti da se ovakvi zahtevi pravilno obrade. Takođe, obezbediti da u slučaju izuzetaka, resursi budu ispravno zatvoreni. \hfill (3p)
\end{itemize}

\vspace{10pt}
\noindent
\begin{tabular}{lll}
  \begin{lstlisting}
  > ins Magnus Carlsen
  ins je uspesno izvrsen
  > sel 1
  Magnus Carlsen: 1300
  > upd 1 30
  upd je uspesno izvrsen
  > sel 1
  Magnus Carlsen: 1330
  > upd 1 -10
  upd je uspesno izvrsen
  > sel 1
  Magnus Carlsen: 1320
  > bye
  \end{lstlisting}&
  \begin{lstlisting}
  > ins Fabiano Caruana
  ins je uspesno izvrsen
  > sel 1
  Fabiano Caruana: 1300
  > upd 1 1500
  upd je uspesno izvrsen
  > sel 1
  Fabiano Caruana: 2800
  > upd 1 -10000
  upd je uspesno izvrsen
  > sel 1
  Fabiano Caruana: 1300
  > bye
  \end{lstlisting}&
  \begin{lstlisting}
  > ins Marko
  ins je uspesno izvrsen
  > sel 1
  Marko: 1300
  // drugi klijent: upd
  > sel 1 
  Marko: 1400
  > ins Marko
  ins je uspesno izvrsen
  // nije isti Marko
  > sel 2
  Marko: 1300
  > bye
  \end{lstlisting}
\end{tabular}
% ----------- END 1 -----------
\vspace{15pt}
\begin{center}
  \textbf{------------------------------------------------------------------------------------------------------------------------------}
\end{center}
\textit{Napomena: Ohrabrujemo studente da koriste \texttt{netcat} kako bi testirali delimi\v{c}ne implementacije i otkrili gre\v{s}ke pre vremena. Takodje, ukoliko se npr. presko\v{c}i implementacija servera, mo\v{z}e se mock-ovati server putem \texttt{netcat}-a.} 
\begin{center}
  \textbf{--------------------------------------------------- Okrenite stranu! ---------------------------------------------------}
\end{center}
\newpage
% ----------- BEGIN 2 -----------
  \item \textbf{UDP Sockets (15p)}
  \\Implementirati UDP server koji klijentima sluzi Fibona\v{c}ijeve brojeve. 
  \begin{itemize}
    \item Napraviti Java aplikaciju koja ima ulogu UDP servera. Slu\v{s}ati na portu 12345 i primiti datagrame od klijenata koriste\'c{}i \texttt{DatagramPacket} klasu. Sadr\v{z}aj svakog datagrama primljenog od klijenta je velicine \texttt{4B}. Ispisati tekst \texttt{Stigao datagram!} kad god server primi validan datagram od nekog klijenta. \hfill (3p)
    \item Napraviti Java aplikaciju koja ima ulogu UDP klijenta. Poslati UDP datagram lokalnom serveru na portu 12345 koriste\'c{}i \texttt{DatagramPacket} klasu. Sadr\v{z}aj datagrama je jedan pozitivan ceo broj veli\v{c}ine \texttt{4B} koji se unosi sa standardnog ulaza. \hfill (3p)
    \item Inicijalno, klijent \v{s}alje serveru datagram sa sadr\v{z}ajem koji predstavlja prirodni broj $n$ (ne ve\'c{}i od $80$). Nakon primanja datagrama, server klijentu \v{s}alje prvih $n$ Fibona\v{c}ijevih brojeva \footnote{Fibona\v{c}ijevi brojevi se ra\v{c}unaju po formuli: $F_0 = 0$, $F_1 = 1$, $F_n = F_{n-1} + F_{n-2}$ za $n > 1$}, po jedan u svakom datagramu. Svaki poslati broj je veli\v{c}ine \texttt{8B}. \hfill (4p)
    \item Primiti datagrame na klijentskoj strani, i za svaki primljeni datagram ispisati primljeni broj na standardni izlaz. \hfill (2p)
    \item Ke\v{s}irati Fibona\v{c}ijeve brojeve na serverskoj strani (ne ra\v{c}unati ih iznova za svakog klijenta). \hfill (2p)
    \item Postarati se da su svi resursi ispravno zatvoreni u slu\v{c}aju izuzetka. \hfill (1p)
  \end{itemize}
% ----------- END 2 -----------
\end{enumerate}
\end{document}
% ----------- END DOCUMENT -----------
