\documentclass[]{article}

% ----------- BEGIN PACKAGES -----------
\usepackage[utf8]{inputenc}
\usepackage[english,serbian]{babel}
\usepackage[margin=0.7in]{geometry}
\usepackage{url}
\usepackage{float}
\usepackage[graphicx]{realboxes}
\usepackage{listings}
\usepackage{textcomp}
\usepackage{xcolor}
\usepackage{titlesec}
\usepackage{adjustbox}
\lstset {
    language=Java,
    frame=none,
    %xleftmargin=-.25in,
    %xrightmargin=.25in
    framesep=10pt,
    tabsize=4,
    showstringspaces=false,
    upquote=true,
    commentstyle=\color{black},
    keywordstyle=\color{black},
    stringstyle=\color{black},
    basicstyle=\small\ttfamily,
    emph={int,char,double,float,unsigned,void,bool},
    emphstyle={\color{black}},
    escapechar=\&,
    classoffset=1,
    morekeywords={>,<,.,;,,,-,!,=,~},
    keywordstyle=\color{black},
    classoffset=0,
    breaklines=true,
    literate=%
         {á}{{\'a}}1
         {í}{{\'i}}1
         {é}{{\'e}}1
         {ý}{{\'y}}1
         {ú}{{\'u}}1
         {ó}{{\'o}}1
         {ě}{{\v{e}}}1
         {š}{{\v{s}}}1
         {č}{{\v{c}}}1
         {ř}{{\v{r}}}1
         {ž}{{\v{z}}}1
         {ď}{{\v{d}}}1
         {ť}{{\v{t}}}1
         {ň}{{\v{n}}}1                
         {ů}{{\r{u}}}1
         {Á}{{\'A}}1
         {Í}{{\'I}}1
         {É}{{\'E}}1
         {Ý}{{\'Y}}1
         {Ú}{{\'U}}1
         {Ó}{{\'O}}1
         {Ě}{{\v{E}}}1
         {Š}{{\v{S}}}1
         {Č}{{\v{C}}}1
         {Ř}{{\v{R}}}1
         {Ž}{{\v{Z}}}1
         {Ď}{{\v{D}}}1
         {Ť}{{\v{T}}}1
         {Ň}{{\v{N}}}1                
         {Ů}{{\r{U}}}1 
}
\pagenumbering{gobble}
% ----------- END PACKAGES -----------

% ----------- BEGIN PREAMBLE -----------
\titlespacing\title{left spacing}{before spacing}{after spacing}[right]

\title{Ra\v{c}unarske mre\v{z}e 4I, Kolokvijum - A}
\date{24.11.2019.}

\begin{document}

\makeatletter
\begin{center}

{\fontsize{12pt}{14pt}\selectfont\bfseries\@title\par}
\@date
\vspace{5mm}

\noindent\fbox{%
    \parbox{\textwidth}{%
      Pro\v{c}itati sve zadatke \textbf{pa\v{z}ljivo} pre rada - sve \v{s}to nije navedeno ne mora da se implementira! 

      Na \texttt{Desktop}-u se nalazi zip arhiva. Unutar arhive se nalazi direktorijum sa imenom \texttt{rm\_kolA\_Ime\_Prezime\_miGGXXX}\\
      u kome se nalazi validan IntelliJ projekat. Izvu\'c{}i direktorijum iz arhive na Desktop i preimenovati ga tako da ime\\
      odgovara podacima studenta. Otvoriti IntelliJ IDEA, izabrati opciju \texttt{Open project} i otvoriti pomenuti direktorijum. Sve kodove ostaviti unutar ve\'c{} kreiranih Java fajlova. \textbf{Kodovi koji se ne prevode se ne\'c{}e pregledati.}\\
      \textbf{Nepo\v{s}tovanje formata ulaza/izlaza nosi kaznu od -10\% poena na zadatku!}
      Vreme za rad: \textbf{2h}.
    }%
}
\end{center}
\makeatother
% ----------- END PREAMBLE -----------

\vspace{5pt}

% ----------- BEGIN DOCUMENT -----------
\begin{enumerate}

% ----------- BEGIN 1 -----------
  \item \textbf{Tokovi podataka i niti (15p)}
  \\Napisati program koji ispisuje ukupan broj pojavljivanja zadatog karaktera u svim tekstualnim fajlovima sa spiska URL-ova. 
  \begin{itemize}
    \item U datoteci \texttt{urls.txt} unutar direktorijuma \texttt{tests} na Desktop-u se nalazi spisak URL-ova (po jedan u svakoj liniji). Koriste\'c{}i odgovaraju\'c{}e \textbf{baferisane} ulazne tokove pro\v{c}itati sadr\v{z}aj pomenutog fajla i ispisati broj linija u tom fajlu. \hfill (2p)
    \item Za svaku pro\v{c}itanu liniju fajla \texttt{urls.txt} kreirati novi URL objekat koriste\'c{}i \texttt{URL} klasu. Presko\v{c}iti sve linije koje ne predstavljaju validan URL. \hfill (1p)
    \item Za svaki validni URL proveriti protokol koji se koristi. Ukoliko je protokol \texttt{FILE} i ukoliko putanja vodi do tekstualnog fajla (ekstenzija \texttt{.txt}), kreirati zasebnu nit koja \'c{}e otvoriti \textbf{baferisani} ulazni tok do tog fajla putem URL klase i pro\v{c}itati sadr\v{z}aj fajla (detalji obrade su u narednoj stavci). Kodnu stranu prilikom u\v{c}itavanja postaviti na UTF-8. Ukoliko fajl na datoj putanji ne postoji, ispisati poruku i ugasiti nit. \hfill (5p)
    \item Pre parsiranja fajla \texttt{urls.txt}, sa standardnog ulaza u\v{c}itati jedan karakter. Prebrojati koliko se puta zadati karakter pojavljuje u svim fajlovima iz prethodne stavke tako \v{s}to \'c{}e svaka nit prebrojati pojavljivanja za fajl koji joj je dodeljen. Ispisati ukupan broj na standardni izlaz (videti primere ispisa ispod teksta zadatka). Pritom, paziti na sinhronizaciju niti ukoliko se koristi deljeni broja\v{c}. \hfill (5p)
    \item Postarati se da program ispravno barata specijalnim slu\v{c}ajevima (npr. ako fajl ne postoji na datoj putanji) i ispravno zatvoriti sve kori\v{s}\'c{}ene resurse u slu\v{c}aju izuzetka. \hfill (2p)
  \end{itemize}

  \noindent
  \begin{lstlisting}
  ulaz:  a
  izlaz: lines:     29
         not found: /home/ispit/Desktop/tests/404.txt
         result:    3915
  \end{lstlisting}
  \begin{lstlisting}
  ulaz:  %
  izlaz: lines:     29
         not found: /home/ispit/Desktop/tests/404.txt
         result:    0
  \end{lstlisting}
  \begin{lstlisting}
  ulaz:  č
  izlaz: lines:     29
         not found: /home/ispit/Desktop/tests/404.txt
         result:    1
  \end{lstlisting}
% ----------- END 1 -----------
\begin{center}
  \vspace{10pt}
  \textbf{--------------------------------------------------- Okrenite stranu! ---------------------------------------------------}
\end{center}
\newpage
% ----------- BEGIN 2 -----------
  \item \textbf{Sockets (15p)}
  \\
  \begin{itemize}
    \item Napraviti Java aplikaciju koja ima ulogu servera. Pokrenuti lokalni server na portu 27182, koristeći \texttt{ServerSocket} klasu. Server za svakog povezanog klijenta pokreće zasebnu nit u kojoj se taj klijent obradjuje. \hfill (4p)
    \item Napraviti Java aplikaciju koja ima ulogu klijenta. Povezati se na lokalni server na portu 27182 koristeći \texttt{Socket} klasu. Nakon formiranja konekcije klijent šalje serveru relativnu putanju do fajla na fajlsistemu servera u odnosu na direktorijum \texttt{tests}, realan broj \texttt{x}, realan broj \texttt{eps} (svi podaci uneti sa standardnog ulaza). Ispisati primljenu putanju na standardni izlaz na serverskoj strani, a klijentu poslati indikator da li je putanja validna ili ne (npr. putanja je nevalidna ukoliko fajl na toj putanji ne postoji). \hfill (3p)
    \item Zadatak servera je da prebroji koliko ima brojeva koji se nalaze u epsilon okolini broja \texttt{x} $(\textit{x}-\textit{eps}, \textit{x}+\textit{eps})$ u fajlu na primljenoj putanji i da taj broj pošalje klijentu. Klijent rešenje ispisuje na standardni izlaz. U slučaju da u fajlu nema realnih brojeva, klijent treba da ispiše poruku \textit{Fajl ne sadrzi realne brojeve} na standardni izlaz. \hfill (7p)
    \item Postarati se da su svi resursi ispravno zatvoreni u slučaju izuzetka. \hfill (1p)
  \end{itemize}

  \noindent
  \begin{lstlisting}
  ulaz:          dir2/1.test 3.5 0.01
  izlaz server:  dir2/1.test
  izlaz klijent: Validna putanja
                 Fajl ne sadrzi realne brojeve
  \end{lstlisting}
  \begin{lstlisting}
  ulaz:          dir2/2.test 3.5 0.5
  izlaz server:  dir2/2.test
  izlaz klijent: Validna putanja
                 5
  \end{lstlisting}
  \begin{lstlisting}
  ulaz:          dir2/2.test 3.5 0.3
  izlaz server:  dir2/2.test
  izlaz klijent: Validna putanja
                 4
  \end{lstlisting}

% ----------- END 2 -----------
\end{enumerate}
\end{document}
% ----------- END DOCUMENT -----------
