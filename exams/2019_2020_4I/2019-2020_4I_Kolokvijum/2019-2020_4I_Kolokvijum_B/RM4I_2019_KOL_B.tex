\documentclass[]{article}

% ----------- BEGIN PACKAGES -----------
\usepackage[utf8]{inputenc}
\usepackage[english,serbian]{babel}
\usepackage[margin=0.7in]{geometry}
\usepackage{url}
\usepackage{float}
\usepackage[graphicx]{realboxes}
\usepackage{listings}
\usepackage{textcomp}
\usepackage{xcolor}
\usepackage{titlesec}
\usepackage{adjustbox}
\lstset {
    language=Java,
    frame=none,
    %xleftmargin=-.25in,
    %xrightmargin=.25in
    framesep=10pt,
    tabsize=4,
    showstringspaces=false,
    upquote=true,
    commentstyle=\color{black},
    keywordstyle=\color{black},
    stringstyle=\color{black},
    basicstyle=\small\ttfamily,
    emph={int,char,double,float,unsigned,void,bool},
    emphstyle={\color{black}},
    escapechar=\&,
    classoffset=1,
    morekeywords={>,<,.,;,,,-,!,=,~},
    keywordstyle=\color{black},
    classoffset=0,
    breaklines=true
}
\pagenumbering{gobble}
% ----------- END PACKAGES -----------

% ----------- BEGIN PREAMBLE -----------
\titlespacing\title{left spacing}{before spacing}{after spacing}[right]

\title{Ra\v{c}unarske mre\v{z}e 4I, Kolokvijum - B}
\date{24.11.2019.}

\begin{document}

\makeatletter
\begin{center}

{\fontsize{12pt}{14pt}\selectfont\bfseries\@title\par}
\@date
\vspace{5mm}

\noindent\fbox{%
    \parbox{\textwidth}{%
      Pro\v{c}itati sve zadatke \textbf{pa\v{z}ljivo} pre rada - sve \v{s}to nije navedeno ne mora da se implementira! 

      Na \texttt{Desktop}-u se nalazi zip arhiva. Unutar arhive se nalazi direktorijum sa imenom \texttt{rm\_kolB\_Ime\_Prezime\_miGGXXX}\\
      u kome se nalazi validan IntelliJ projekat. Izvu\'c{}i direktorijum iz arhive na Desktop i preimenovati ga tako da ime\\
      odgovara podacima studenta. Otvoriti IntelliJ IDEA, izabrati opciju \texttt{Open project} i otvoriti pomenuti direktorijum. Sve kodove ostaviti unutar ve\'c{} kreiranih Java fajlova. \textbf{Kodovi koji se ne prevode se ne\'c{}e pregledati.}\\
      \textbf{Nepo\v{s}tovanje formata ulaza/izlaza nosi kaznu od -10\% poena na zadatku!}
      Vreme za rad: \textbf{2h}.
    }%
}
\end{center}
\makeatother
% ----------- END PREAMBLE -----------

\vspace{5pt}

% ----------- BEGIN DOCUMENT -----------
\begin{enumerate}

% ----------- BEGIN 1 -----------
  \item \textbf{Tokovi podataka i niti (15p)}
  \\Napisati program koji rekurzivno obilazi direktorijum i ispisuje broj linija u svim \texttt{.c} fajlovima koriste\'c{}i niti.
  \begin{itemize}
    \item Na Desktop-u se nalazi direktorijum \texttt{tests}. Obi\'c{}i pomenuti direktorijum i ispisati na standardni izlaz ukupan broj svih regularnih fajlova unutar tog direktorijuma. \hfill (2p)
    \item Za svaki pronadjeni fajl sa ekstenzijom \texttt{.c} kreirati novi URL objekat koriste\'c{}i \texttt{URL} klasu i \texttt{FILE} protokol. 
    Ispisati kreirane URL-ove na standardni izlaz (videti primere ispod). 
    \vspace{5pt}
    \\
    \begin{tabular}{ll}
      Putanja & Odgovaraju\'c{}i URL\\
      \texttt{tests/dir1/1.txt} & \texttt{FILE:///home/ispit/Desktop/tests/dir1/1.txt}
    \end{tabular}
    \vspace{5pt}
    \hfill (2p)
    \item Za svaki kreirani URL, kreirati zasebnu nit koja \'c{}e otvoriti \textbf{baferisani} ulazni tok do resursa putem URL klase i pro\v{c}itati sadr\v{z}aj fajla (detalji obrade su u narednoj stavci). Kodnu stranu prilikom u\v{c}itavanja postaviti na ASCII. \hfill (4p)
    \item Na standardni izlaz ispisati ukupan broj linija u svim fajlovima iz prethodne stavke tako \v{s}to \'c{}e svaka nit prebrojati linije za fajl koji joj je dodeljen (videti primer ispisa ispod teksta zadatka). Pritom, paziti na sinhronizaciju niti ukoliko se koristi deljeni broja\v{c}. \hfill (5p)
    \item Postarati se da program u slu\v{c}ajevima ispravno zatvori sve kori\v{s}\'c{}ene resurse. \hfill (1p)
  \end{itemize}

  \noindent
  \begin{lstlisting}
  ulaz:  
  izlaz: files:     20
         url:       FILE:///home/ispit/Desktop/tests/dir/dir1/dir11/palindrome.c
         url:       FILE:///home/ispit/Desktop/tests/dir/dir1/smile.c
         url:       FILE:///home/ispit/Desktop/tests/dir/dir2/dir21/pi.c
         result:    85
  \end{lstlisting}
% ----------- END 1 -----------
\begin{center}
  \vspace{10pt}
  \textbf{--------------------------------------------------- Okrenite stranu! ---------------------------------------------------}
\end{center}
\newpage
% ----------- BEGIN 2 -----------
  \item \textbf{Sockets (15p)}
  \\
  \begin{itemize}
    \item Napraviti Java aplikaciju koja ima ulogu servera. Pokrenuti lokalni server na portu 31415, koristeći \texttt{ServerSocket} klasu. Server za svakog povezanog klijenta pokreće zasebnu nit. \hfill (4p)
    \item Napraviti Java aplikaciju koja ima ulogu klijenta. Povezati se na lokalni server na portu 31415 koristeći \texttt{Socket} klasu. Nakon formiranja konekcije klijent šalje serveru relativnu putanju do fajla na fajlsistemu servera (unetu sa standardnog ulaza) u odnosu na direktorijum \texttt{tests}. Ispisati primljenu putanju na standardni izlaz na serverskoj strani, a klijentu poslati indikator da li je putanja validna ili ne (npr. putanja je nevalidna ukoliko fajl na toj putanji ne postoji). \hfill (3p)
    \item Zadatak servera je da izračuna zbir svih realnih brojeva koji se nalaze u datom fajlu. Server šalje klijentu zbir, a klijent rešenje ispisuje na standardni izlaz. U slučaju da u fajlu nema realnih brojeva, klijent treba da ispiše poruku \textit{Fajl ne sadrzi realne brojeve} na standardni izlaz. \hfill (7p)
    \item Postarati se da su svi resursi ispravno zatvoreni u slučaju izuzetka. \hfill (1p)
  \end{itemize}

  \noindent
  \begin{lstlisting}
  ulaz:          dir2/1.test
  izlaz server:  dir2/1.test
  izlaz klijent: Validna putanja
                 Fajl ne sadrzi realne brojeve
  \end{lstlisting}
  \begin{lstlisting}
  ulaz:          dir2/2.test
  izlaz server:  dir2/2.test
  izlaz klijent: Validna putanja
                 69.41
  \end{lstlisting}
  \begin{lstlisting}
  ulaz:          dir2/3.test
  izlaz server:  dir2/3.test
  izlaz klijent: Validna putanja
                 34.27
  \end{lstlisting}
% ----------- END 2 -----------
\end{enumerate}
\end{document}
% ----------- END DOCUMENT -----------

