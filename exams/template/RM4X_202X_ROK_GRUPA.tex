\documentclass[]{article}

% ----------- BEGIN PACKAGES -----------
\usepackage[utf8]{inputenc}
\usepackage[english,serbian]{babel}
\usepackage[margin=0.7in]{geometry}
\usepackage{url}
\usepackage{float}
\usepackage[graphicx]{realboxes}
\usepackage{listings}
\usepackage{textcomp}
\usepackage{xcolor}
\usepackage{titlesec}
\usepackage{adjustbox}
\lstset {
    language=Java,
    frame=none,
    %xleftmargin=-.25in,
    %xrightmargin=.25in
    framesep=10pt,
    tabsize=4,
    showstringspaces=false,
    upquote=true,
    commentstyle=\color{black},
    keywordstyle=\color{black},
    stringstyle=\color{black},
    basicstyle=\small\ttfamily,
    emph={int,char,double,float,unsigned,void,bool},
    emphstyle={\color{black}},
    escapechar=\&,
    classoffset=1,
    morekeywords={>,<,.,;,,,-,!,=,~},
    keywordstyle=\color{black},
    classoffset=0,
    breaklines=true
}
\pagenumbering{gobble}
% ----------- END PACKAGES -----------

% ----------- BEGIN PREAMBLE -----------
\titlespacing\title{left spacing}{before spacing}{after spacing}[right]

\title{Ra\v{c}unarske mre\v{z}e 4X, Ispit - ROK / Kolokvijum - Grupa? GODINA}
\date{}

\begin{document}

\makeatletter
\begin{center}

{\fontsize{12pt}{14pt}\selectfont\bfseries\@title\par}
\@date
\vspace{5mm}

\noindent\fbox{%
    \parbox{\textwidth}{%
      Pro\v{c}itati sve zadatke \textbf{pa\v{z}ljivo} pre rada - sve \v{s}to nije navedeno ne mora da se implementira!\\
      Na \texttt{Desktop}-u se nalazi zip arhiva. Unutar arhive se nalazi IntelliJ projekat u formatu \texttt{rm\_rok\_Ime\_Prezime\_mXGGXXX}.\\ Dekompresovati arhivu na Desktop i ubaciti svoje podatke u ime pomenutog direktorijuma.\\
      Otvoriti IntelliJ IDEA, izabrati opciju \texttt{Open project} (\textbf{ne} \texttt{Import project}!) i otvoriti pomenuti \textbf{direktorijum}.\\ 
      Sve kodove ostaviti unutar ve\'c{} kreiranih Java fajlova. \textbf{Kodovi koji se ne prevode se ne\'c{}e pregledati.}\\
      \textbf{Nepo\v{s}tovanje formata ulaza/izlaza nosi kaznu od -10\% poena na zadatku!}
    }%
}
\end{center}
\makeatother
% ----------- END PREAMBLE -----------

\vspace{5pt}

% ----------- BEGIN DOCUMENT -----------
\begin{enumerate}

% ----------- BEGIN 1 -----------
  \item \textbf{NASLOV ZADATKA 1 (Xp)}
  \\Opis zadatka 1
  \begin{itemize}
    \item Stavka. \hfill (Xp)
    \item Stavka. \hfill (Xp)
    \item Stavka. \hfill (Xp)
    \item Stavka. \hfill (Xp)
    \item Stavka. \hfill (Xp)
  \end{itemize}

  \noindent
  \begin{tabular}{lll}
  \begin{lstlisting}
  ulaz:  a
  izlaz: 123456
  \end{lstlisting}&
  \begin{lstlisting}
  ulaz:  abcdef
  izlaz: 123456
  \end{lstlisting}&
  \begin{lstlisting}
  ulaz:  abcdef
  izlaz: 123456
  \end{lstlisting}
  \end{tabular} 
% ----------- END 1 -----------
\vspace{15pt}
\begin{center}
  \textbf{------------------------------------------------------------------------------------------------------------------------------}
\end{center}
\textit{Napomena: Ohrabrujemo studente da koriste \texttt{netcat} kako bi testirali delimi\v{c}ne implementacije i otkrili gre\v{s}ke pre vremena. Takodje, ukoliko se npr. presko\v{c}i implementacija servera, mo\v{z}e se mock-ovati server putem \texttt{netcat}-a.} 
\begin{center}
  \textbf{--------------------------------------------------- Okrenite stranu! ---------------------------------------------------}
\end{center}
\newpage
% ----------- BEGIN 2 -----------
  \item \textbf{NASLOV ZADATKA 2 (Xp)}
  \\Opis zadatka 2
  \begin{itemize}
    \item Stavka. \hfill (Xp)
    \item Stavka. \hfill (Xp)
    \item Stavka. \hfill (Xp)
    \item Stavka. \hfill (Xp)
    \item Stavka. \hfill (Xp)
  \end{itemize}
  
  \noindent
  \begin{tabular}{lll}
  \begin{lstlisting}
  ulaz:  a
  izlaz: 123456
  \end{lstlisting}&
  \begin{lstlisting}
  ulaz:  abcdef
  izlaz: 123456
  \end{lstlisting}&
  \begin{lstlisting}
  ulaz:  abcdef
  izlaz: 123456
  \end{lstlisting}
  \end{tabular}
% ----------- END 2 -----------
\end{enumerate}
\end{document}
% ----------- END DOCUMENT -----------
