\documentclass[]{article}

% ----------- BEGIN PACKAGES -----------
\usepackage[utf8]{inputenc}
\usepackage[english,serbian]{babel}
\usepackage[margin=0.7in]{geometry}
\usepackage{url}
\usepackage{float}
\usepackage[graphicx]{realboxes}
\usepackage{listings}
\usepackage{textcomp}
\usepackage{xcolor}
\usepackage{titlesec}
\usepackage{adjustbox}
\lstset {
    language=Java,
    frame=none,
    %xleftmargin=-.25in,
    %xrightmargin=.25in
    framesep=10pt,
    tabsize=4,
    showstringspaces=false,
    upquote=true,
    commentstyle=\color{black},
    keywordstyle=\color{black},
    stringstyle=\color{black},
    basicstyle=\small\ttfamily,
    emph={int,char,double,float,unsigned,void,bool},
    emphstyle={\color{black}},
    escapechar=\&,
    classoffset=1,
    morekeywords={>,<,.,;,,,-,!,=,~},
    keywordstyle=\color{black},
    classoffset=0,
    breaklines=true
}
\pagenumbering{gobble}
% ----------- END PACKAGES -----------

% ----------- BEGIN PREAMBLE -----------
\titlespacing\title{left spacing}{before spacing}{after spacing}[right]

\title{Ra\v{c}unarske mre\v{z}e 4R, Ispit - JUN1}
\date{18.06.2020.}

\begin{document}

\makeatletter
\begin{center}

{\fontsize{12pt}{14pt}\selectfont\bfseries\@title\par}
\@date
\vspace{5mm}

\noindent\fbox{%
    \parbox{\textwidth}{%
      Pro\v{c}itati sve zadatke \textbf{pa\v{z}ljivo} pre rada - sve \v{s}to nije navedeno ne mora da se implementira! 

      Na \texttt{Desktop}-u se nalazi zip arhiva. Unutar arhive se nalazi direktorijum u formatu \texttt{rm\_rok\_Ime\_Prezime\_miGGXXX}\\
      u kome se nalazi validan IntelliJ projekat. Izvu\'c{}i direktorijum iz arhive na Desktop i ubaciti svoje podatke u ime.\\
      Otvoriti IntelliJ IDEA, izabrati opciju \texttt{Open project} (ne \texttt{Import project}!) i otvoriti pomenuti direktorijum.\\ 
      Sve kodove ostaviti unutar ve\'c{} kreiranih Java fajlova. \textbf{Kodovi koji se ne prevode se ne\'c{}e pregledati.}\\
      \textbf{Nepo\v{s}tovanje formata ulaza/izlaza nosi kaznu od -10\% poena na zadatku!}
      Vreme za rad: \textbf{3h}.
    }%
}
\end{center}
\makeatother
% ----------- END PREAMBLE -----------

\vspace{5pt}

% ----------- BEGIN DOCUMENT -----------
\begin{enumerate}

% ----------- BEGIN 1 -----------
  \item \textbf{Statistika protokola (15p) (za studente koji nisu radili projekat)}
  \\Napisati program koji vodi statistiku protokola u URL-ovima koji se nalaze u datotekama na fajl sistemu.
  \begin{itemize}
    \item U direktorijumu \texttt{urls} unutar direktorijuma \texttt{tests} na Desktop-u se nalaze datoteke koje sadr\v{z}e spisak URL-ova (po jedan u svakoj liniji). Kori\v{s}\'cenjem odgovaraju\'cih tokova podataka prebrojati koliko linija sadr\v{z}i svaka datoteka. Na standardni izlaz, za svaku datoteku, ispisati njenu apsolutnu putanju i broj linija kao u primeru ispod u levoj koloni. \hfill (2p)
    \item Za svaku datoteku pokrenuti zasebnu nit koja \'ce obradjivati i ispisivati statistiku za tu datoteku. \hfill (2p)
    \item Za svaku pro\v{c}itanu liniju datoteke kreirati novi URL objekat koriste\'c{}i \texttt{URL} klasu. Presko\v{c}iti sve linije koje ne predstavljaju validan URL. \hfill (1p)
    \item Za svaki validni URL proveriti protokol koji se koristi i pratiti ukupan broj pojavljivanja tog protokola u trenutnoj datoteci. Nakon \v{s}to nit obradi datoteku, na standardni izlaz ispisati statistiku kao u primeru ispod u desnoj koloni. \hfill (4p)
    \item Postarati se da se ispisi svake niti na standardni izlaz ne prepliću. \hfill (4p)
    \item Postarati se da program ispravno obradjuje specijalnim slu\v{c}ajeve (npr. ako datoteka ne postoji na datoj putanji) i ispravno zatvoriti sve kori\v{s}\'c{}ene resurse u slu\v{c}aju izuzetka. \hfill (2p)
  \end{itemize}
  
\noindent
\begin{tabular}{ll}
\begin{lstlisting}
/home/ispit/Desktop/tests/urls/1.txt 6
/home/ispit/Desktop/tests/urls/2.txt 4
/home/ispit/Desktop/tests/urls/3.txt 3
/home/ispit/Desktop/tests/urls/4.txt 5
\end{lstlisting}&
\begin{lstlisting}
- /home/ispit/Desktop/tests/urls/4.txt
- /home/ispit/Desktop/tests/urls/3.txt
  http   : 2
  https  : 1
- /home/ispit/Desktop/tests/urls/1.txt
  http   : 4
  https  : 2
- /home/ispit/Desktop/tests/urls/2.txt
  ftp    : 3
  http   : 1
\end{lstlisting}
\end{tabular}
% ----------- END 1 -----------
\vspace{15pt}

\begin{center}
  \textbf{------------------------------------------------------------------------------------------------------------------------------}
\end{center}
\textit{Napomena: Ohrabrujemo studente da koriste \texttt{netcat} kako bi testirali delimi\v{c}ne implementacije i otkrili gre\v{s}ke pre vremena. Takodje, ukoliko se npr. presko\v{c}i implementacija servera, mo\v{z}e se mock-ovati server putem \texttt{netcat}-a.} 
\begin{center}
  \textbf{--------------------------------------------------- Okrenite stranu! ---------------------------------------------------}
\end{center}

\newpage

% ----------- BEGIN 2 -----------
\item \textbf{Zbir brojeva (20p/12p)}
\\Napraviti klijent-server TCP aplikaciju preko koje se ra\v{c}una zbir realnih brojeva.
\begin{itemize}
  \item Napraviti Java klasu koja ima ulogu lokalnog TCP servera, koji oslu\v{s}kuje na portu $10000$, koriste\'c{}i Java \texttt{Socket} API. Za svakog povezanog klijenta, server pokre\'ce zasebnu nit, preko koje se vr\v{s}i komunikacija sa klijentom. \hfill (5p/3p)
  \item Napraviti Java klasu koja ima ulogu klijenta. Klijent formira konekciju sa lokanim serverom na portu $10000$ i \v{s}alje serveru realne brojeve koje \v{c}ita sa standardnog ulaza sve dok korisnik ne unese nisku \texttt{KRAJ}. Server ispisuje svaki realni broj koji dobije od klijenta na standardni izlaz. \hfill (5p/3p)
  \item Server za svakog klijenta ra\v{c}una zbir do tada poslatih realnih brojeva. Nakon \v{s}to klijent po\v{s}alje sve realne brojeve, server klijentu vra\'ca njihov zbir. \hfill (5p/3p)
  \item Klijent na stanadrni izlaz ispisuje realan broj koji mu je poslao server.\hfill (2p/1p)
  \item Postarati se da su svi resursi ispravno zatvoreni u slu\v{c}aju izuzetka. \hfill (3p/2p)
\end{itemize}
% ----------- END 2 -----------

\vspace{15pt}

% ----------- BEGIN 3 -----------
\item \textbf{Povr\v{s}ina kruga (25p/18p)}
\\Napraviti klijent-server TCP aplikaciju preko koje se ra\v{c}unaju povr\v{s}ine krugova datih polupre\v{c}nika.
\begin{itemize}
  \item Napraviti Java klasu koja ima ulogu lokalnog \textbf{neblokiraju\'c{}eg} TCP servera koji oslu\v{s}kuje na portu $31415$, koriste\'c{}i Java \texttt{Channels} API. Nakon konkecije, klijent \v{s}alje serveru realane brojeve koji predstavljaju polupre\v{c}nike krugova. Kao odgovor, server \v{s}alje klijentu po jedan realan broj nakon svakog primljenog polupre\v{c}nika koji predstavlja povr\v{s}inu kruga sa primljenim polupre\v{c}nikom. Ukoliko server primi negativan broj od klijenta, to se tuma\v{c}i kao zahtev za prekid veze. \hfill (15p/10p)
  \item Napraviti Java klasu koja ima ulogu \textbf{blokiraju\'c{}eg} TCP klijenta koriste\'c{}i Java \texttt{Channels} API. Klijent formira konekciju sa lokanim serverom na portu $31415$. Nakon formiranja konekcije, klijent serveru \v{s}alje jedan po jedan realan broj u\v{c}itan sa standardnog ulaza i pre u\v{c}itavanja narednog broja \v{c}eka odgovor od servera. Ispisati odgovore servera na standardni izlaz. \hfill (8p/6p)
  \item Postarati se da su svi resursi ispravno zatvoreni u slu\v{c}aju izuzetka. \hfill (2p)
\end{itemize}
% ----------- END 3 -----------

\end{enumerate}

\end{document}
% ----------- END DOCUMENT -----------
